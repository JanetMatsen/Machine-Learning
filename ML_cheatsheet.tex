\documentclass[10pt,landscape]{article}

\usepackage{multicol}
\usepackage{calc}
\usepackage{ifthen}
\usepackage[landscape]{geometry}
\usepackage{graphicx}
\usepackage{amsmath, amssymb, amsthm}
\usepackage{latexsym, marvosym}
\usepackage{pifont}
\usepackage{lscape}
\usepackage{graphicx}
\usepackage{array}
\usepackage{booktabs}
\usepackage{bm}  % bold math \bm{}
\usepackage[bottom]{footmisc}
\usepackage{tikz}
\usetikzlibrary{shapes}
\usepackage{pdfpages}
\usepackage{wrapfig}
\usepackage{enumitem}
\setlist[description]{leftmargin=0pt}
\usepackage{xfrac}
\usepackage[pdftex,
            pdfauthor={Janet Matsen},
            pdftitle={Machine Learning Cheatsheet},
            pdfsubject={Notes from UW CSE 446 Winter 2016},
            pdfkeywords={machine learning} {statistics} {cheatsheet} {pdf} {cheat} {sheet} {formulas} {equations}
            ]{hyperref}
\usepackage{relsize}
\usepackage{rotating}


 \newcommand\independent{\protect\mathpalette{\protect\independenT}{\perp}}
    \def\independenT#1#2{\mathrel{\setbox0\hbox{$#1#2$}%
    \copy0\kern-\wd0\mkern4mu\box0}} 
            
% Janet defined
\DeclareMathOperator*{\argmin}{arg\,min}
\DeclareMathOperator*{\argmax}{arg\,max}

% Probably from Stat cheatsheet:            
\newcommand{\noin}{\noindent}    
\newcommand{\logit}{\textrm{logit}} 
\newcommand{\var}{\textrm{Var}}
\newcommand{\cov}{\textrm{Cov}} 
\newcommand{\corr}{\textrm{Corr}} 
\newcommand{\N}{\mathcal{N}}
\newcommand{\Bern}{\textrm{Bern}}
\newcommand{\Bin}{\textrm{Bin}}
\newcommand{\Beta}{\textrm{Beta}}
\newcommand{\Gam}{\textrm{Gamma}}
\newcommand{\Expo}{\textrm{Expo}}
\newcommand{\Pois}{\textrm{Pois}}
\newcommand{\Unif}{\textrm{Unif}}
\newcommand{\Geom}{\textrm{Geom}}
\newcommand{\NBin}{\textrm{NBin}}
\newcommand{\Hypergeometric}{\textrm{HGeom}}
\newcommand{\HGeom}{\textrm{HGeom}}
\newcommand{\Mult}{\textrm{Mult}}

\geometry{top=.4in,left=.2in,right=.2in,bottom=.4in}

\pagestyle{empty}
\makeatletter
\renewcommand{\section}{\@startsection{section}{1}{0mm}%
                                {-1ex plus -.5ex minus -.2ex}%
                                {0.5ex plus .2ex}%x
                                {\normalfont\large\bfseries}}
\renewcommand{\subsection}{\@startsection{subsection}{2}{0mm}%
                                {-1explus -.5ex minus -.2ex}%
                                {0.5ex plus .2ex}%
                                {\normalfont\normalsize\bfseries}}
\renewcommand{\subsubsection}{\@startsection{subsubsection}{3}{0mm}%
                                {-1ex plus -.5ex minus -.2ex}%
                                {1ex plus .2ex}%
                                {\normalfont\small\bfseries}}
\makeatother

\setcounter{secnumdepth}{0}

\setlength{\parindent}{0pt}
\setlength{\parskip}{0pt plus 0.5ex}

% -----------------------------------------------------------------------

\usepackage{titlesec}

\titleformat{\section}
{\color{blue}\normalfont\large\bfseries}
{\color{blue}\thesection}{1em}{}
\titleformat{\subsection}
{\color{cyan}\normalfont\normalsize\bfseries}
{\color{cyan}\thesection}{1em}{}
% Comment out the above 5 lines for black and white

\begin{document}

\raggedright
\footnotesize
\begin{multicols*}{3}

% multicol parameters
% These lengths are set only within the two main columns
%\setlength{\columnseprule}{0.25pt}
\setlength{\premulticols}{1pt}
\setlength{\postmulticols}{1pt}
\setlength{\multicolsep}{1pt}
\setlength{\columnsep}{2pt}

%%%%%%%%%%%%%%%%%%%%%%%%%%%%%%%%%%%%
%%% TITLE
%%%%%%%%%%%%%%%%%%%%%%%%%%%%%%%%%%%%

\begin{center}
    {\color{blue} \Large{\textbf{Machine Learning Cheatsheet}}} \\
   % {\Large{\textbf{Probability Cheatsheet}}} \\
    % comment out line with \color{blue} and uncomment above line for b&w
\end{center}

%%%%%%%%%%%%%%%%%%%%%%%%%%%%%%%%%%%%
%%% ATTRIBUTIONS
%%%%%%%%%%%%%%%%%%%%%%%%%%%%%%%%%%%%

\scriptsize

Janet Matsen's Machine Learning (ML) notes from CSE 446, Winter 2016.  \url{http://courses.cs.washington.edu/courses/cse446/16wi/}

Used LaTeX template from an existing Statistics cheat sheet: \url{https://github.com/wzchen/probability_cheatsheet}, by William Chen (\url{http://wzchen.com}) and Joe Blitzstein. 

% Licensed under \texttt{\href{http://creativecommons.org/licenses/by-nc-sa/4.0/}{CC BY-NC-SA 4.0}}. 
% I have not asked for the rights to images used in lecture. 

\begin{center}
    Last Updated \today
\end{center}

% Cheatsheet format from
% http://www.stdout.org/$\sim$winston/latex/

%%%%%%%%%%%%%%%%%%%%%%%%%%%%%%%%%%%%
%%% BEGIN CHEATSHEET
%%%%%%%%%%%%%%%%%%%%%%%%%%%%%%%%%%%%
  
    \hfill \\   
     \hfill \\  
\smallskip \hrule height 2pt \smallskip

\section{Essential ML ideas}
\smallskip \hrule height 2pt \smallskip

\begin{itemize}
	\item Never ever \underline{ever} touch the test set
	\item You know you are overfitting when there is a big test between train and test results.  E.g. metric of percent wrong. 
	\item Need to be comfortable taking a hit on fitting accuracy if you can get a benefit on the result.
	\item Bias vs variance trade-off.  
		High bias when the model is too simple \& doesn't fit the data well.  
		High variance is when small changes to the data set lead to large solution changes. 
	\item If features are non discriminative in the beginning, they don't work for any classifier.  % week 4 reminder
	\item Your feature vector often has a smaller dimension that the feature space.    % week 4, Friday. 
		If you have too long of a feature vector, you may get overfitting. 

\end{itemize}

\section{Math/Stat Review}
\smallskip \hrule height 2pt \smallskip

\begin{description}
        \item[Random Variable X] belongs to set $\Omega$  
        \item[Conditional Probability \emph{is} Probability]  $P({A}|{ B})$ is a probability function for any fixed $B$. Any theorem that holds for probability also holds for conditional probability.   $P({A}|{ B}) = P(A \cap B)/P(B)$
        \item[Bayes' Rule] - Bayes' Rule unites marginal, joint, and conditional probabilities. We use this as the definition of conditional probability. 
        		\[P({\bf A}|{\bf B}) = \frac{P({\bf A} \cap {\bf B})}{P({\bf B})} = \frac{P({\bf B}|{\bf A})P({\bf A})}{P({\bf B})}\] 
		\[P(A = a \mid B) = \frac{P(A=a) P(B \mid A=a)}{\sum\limits_{a'} P(A=a) P(B \mid A=a)} \]   % TA lecture 1/7/2015
        \item[Law of Total Probability]: $\sum\limits_x P(X=x) = 1$
        \item[Product Rule]: $P(A,B) = P(A \mid B) \cdot P(B)$  % TA lecture 1/7/2015
        \item[Sum Rule]: $P(A) = \sum\limits_{x \in \Omega} P(A, B=b)$  % TA lecture 1/7/2015
        \item[i.i.d]: $D=\{x_i | i=1 \dots n\}, P(D | \theta) = \prod_i P(x_i \mid \theta)$
\end{description}

Vocab:
\begin{itemize}
	\item \textbf{likelihood function} $L(\theta | O)$ is called as the likelihood function. $\theta$ = unknown parameters, $O$ is the observed outcomes.  The likelihood function is conditioned on the observed $O$ and that it is a function of the unknown parameters $\theta$.  Not a probability density function.
	\item \textbf{"likelihood" vs "probability"}: if discrete, $L(\theta | O)= P(O | \theta)$.  If continuous, $P(O|\theta)=0$ so instead we estimate $\theta$ given $O$ by maximizing $L(\theta | O)= f(O | \theta)$ where $f$ is the pdf associated with the outcomes $O$. 
		% http://stats.stackexchange.com/questions/2641/what-is-the-difference-between-likelihood-and-probability
	\item \textbf{hypothesis space}
\end{itemize} 

\subsection{Law of Total Probability (LOTP)}  % from cheat sheet: https://github.com/wzchen/probability_cheatsheet/blob/master/probability_cheatsheet.tex
Let ${ B}_1, { B}_2, { B}_3, ... { B}_n$ be a \emph{partition} of the sample space (i.e., they are disjoint and their union is the entire sample space).
\begin{align*} 
    P({ A}) &= P({ A} | { B}_1)P({ B}_1) + P({ A} | { B}_2)P({ B}_2) + \dots + P({ A} | { B}_n)P({ B}_n)\\
    P({ A}) &= P({ A} \cap { B}_1)+ P({ A} \cap { B}_2)+ \dots + P({ A} \cap { B}_n)
    \end{align*} 
    For \textbf{LOTP with extra conditioning}, just add in another event $C$!
    \begin{align*} 
    P({ A}| { C}) &= P({ A} | { B}_1, { C})P({ B}_1 | { C}) + \dots +  P({ A} | { B}_n, { C})P({ B}_n | { C})\\
    P({ A}| { C}) &= P({ A} \cap { B}_1 | { C})+ P({ A} \cap { B}_2 | { C})+ \dots +  P({ A} \cap { B}_n | { C})
\end{align*} 

Special case of LOTP with ${ B}$ and ${ B^c}$ as partition:
   \begin{align*} 
P({ A}) &= P({ A} | { B})P({ B}) + P({ A} | { B^c})P({ B^c}) \\
P({ A}) &= P({ A} \cap { B})+ P({ A} \cap { B^c}) \\
   \end{align*} 
   
\subsection{Bayes' Rule}  % from cheat sheet: https://github.com/wzchen/probability_cheatsheet/blob/master/probability_cheatsheet.tex

\textbf{Bayes' Rule, and with extra conditioning (just add in $C$!)}
         \[P({ A}|{ B})  = \frac{P({ B}|{ A})P({ A})}{P({ B})}\]
         \[P({ A}|{ B}, { C}) = \frac{P({ B}|{ A}, { C})P({ A} | { C})}{P({ B} | { C})}\]
         We can also write
         $$P(A|B,C) = \frac{P(A,B,C)}{P(B,C)} = \frac{P(B,C|A)P(A)}{P(B,C)}$$
\textbf{Odds Form of Bayes' Rule}
\[\frac{P({ A}| { B})}{P({ A^c}| { B})} = \frac{P({ B}|{ A})}{P({ B}| { A^c})}\frac{P({ A})}{P({ A^c})}\]
The \emph{posterior odds} of $A$ are the \emph{likelihood ratio} times the \emph{prior odds}. 
\hfill \\ \hfill \\

Practice:  What is $P(disease \mid + test)$ if P(disease) = 0.01, \hfill \\  % TA lecture 1/7/2015
  P(+ $\mid$ disease) = 0.99, P(+ $\mid$ no disease) = 0.01? 
% ANS: P(disease | +) = P(-d)*P(+ | d) / (P(-d)*P(+|-d) + P(d)*P(+ | -d)

\subsection{Expectation}  % TA lecture 1/7/2015
\begin{description}
        \item[f(X)] probability distribution function of X  % TA lecture 1/7/2015
        \item[X $\sim$ P]: X is distributed according to P.   % TA lecture 1/7/2015
        \item[Expected value of f under P]: $E_{P}[f(x)] = \sum\limits_{x} p(x)f(x)$
\end{description} 

E.g. unbiased coin.  x = {1, 2, 3, 4, 5, 6}.  p(X=x) = 1/6 for all x.  \hfill \\
E(X) = $\sum\limits_{x} p(x) \cdot x = (1/6) \cdot [1 + 2 + 3 + 4 + 5 + 6] = 3.5$

\subsection{Entropy}
Always greater than  or equal to 0.  Zero when outcome is certain.  1 for uniform distribution. \hfill \\
Entropy is based on a pdf, not a list of labels. E.g.  \hfill \\
H[1,1,0] $\rightarrow$ H[2/2, 1/3]. 

$X \sim P$, $x \in \Omega$  \hfill \\

\hfill \\
First define \textbf{Surprise}: $S(x) = -\log_2 p(x)$   \hfill \\
$S(X = \mbox{heads}) = -\log_2 (1/2) = 1 $.     \hfill \\
\begin{description}  % http://www.cs.cmu.edu/~venkatg/teaching/ITCS-spr2013/notes/15359-2009-lecture25.pdf
        \item[Axiom 1]: S(1) = 0. (If an event with probability 1 occurs, it is not surprising at all.)  
        \item[Axiom 2]: S(q) $>$ S(p) if q $<$ p. (When more unlikely outcomes occur, it is more surprising.)  
        \item[Axiom 3]: S(p) is a continuous function of p. (If an outcome�s probability changes by a tiny
amount, the corresponding surprise should not change by a big amount.)
	\item[Axiom 4]: S(pq) = S(p) $+$ S(q). (Surprise is additive for independent outcomes.)
\end{description}
Surprise of 7 = pretty surprised.  Probability of $1/2^7$ of happening
\hfill \\ 

(Shannon) \textbf{Entropy}:   
% http://www.cs.cmu.edu/~venkatg/teaching/ITCS-spr2013/notes/15359-2009-lecture25.pdf
\begin{align*}
	H[X] &= - \sum\limits_x p(x) \cdot \log_2 p(x) \\
		&= - \sum\limits_x p(x) S(x)  \\
		&= E[S(x)]  
\end{align*}
The entropy is the expectation of the surprise.  Throw out x for $p(x)=0$ because log(0) is $\infty$. \hfill \\
\hfill \\
\underline{Binary Entropy Function}:  $p(X = 1) = \theta$ and $p(X = 0) = 1 - \theta$
\begin{align*}
	H(X) &= - [p(X=1) \log_2 p(X=1)+p(X=0) \log_2 p(X=0)]  \\
		& = - [\theta \log_2 \theta+(1 - \theta) \log_2(1 - \theta)]
\end{align*}
\hfill \\

\underline{Entropy of an unbiased coin flip:} \hfill \\
X is a coin flip. $P(X=\mbox{heads}) = 1/2$, $P(X=\mbox{tails}) = 1/2$  \hfill \\
Note: $\log_2(1/2) = -1$, $- \log_2(1/2) = \log_2(2) = 1$   \hfill \\
$H[X] = -[1/2 \log_2(1/2) + 1/2 \log_2(1/2)] = 1$   \hfill \\
\hfill \\
\underline{Entropy of a coin that always flips to heads:} \hfill \\
$P(X=\mbox{heads}) = 1$, $P(X=\mbox{tails}) = 0$  \hfill \\
Note: $\log_x(0) = 0$   \hfill \\
$H[X] = -[1 \log_2(1) + 0] = 0$   \hfill \\
No surprise: you are sure what you are going to get.  \hfill \\
 \hfill \\

Binary entropy plot. 
\begin{minipage}{\linewidth}
\begin{center}
\includegraphics[width=1.3in]{figures/Binary_entropy_plot.pdf}
\end{center}
\end{minipage}

\underline{Canonical example:} \hfill \\  % TA lecture 1/7/2015. 
\begin{tabular}{ l | r }
  X & Y \\ \hline
  0 & 1 \\
  1 & 0 \\
  1 & 1 \\
\end{tabular}
\hfill \\
If you want to estimate entropy of X, you can use P(X=0).  \hfill \\
\begin{align*}
	H[X] &= -[\frac{1}{3} \log_2 \frac{1}{3} + \frac{2}{3} \log_2 \frac{2}{3}] \\
		& = \frac{1}{3} \log_2 3 + \frac{2}{3} \log_2 3 - \frac{2}{3} \log_2 2 \\
		&= \log_2 3 - \frac{2}{3} \approx 0.91
\end{align*}
This time H[X] = H[Y] because of symmetry.  \hfill \\

The discrete distribution with maximum entropy is the uniform distribution.  For K values of X, $H(X) = \log_2 K$ \hfill \\  % book pg 57
Conversely, the distribution with minimum entropy (which is zero) is any delta-function that puts all its mass on one state. Such a distribution has no uncertainty. 
\hfill \\

\subsection{Conditional Entropy}
If you don't know x:  (this is kind of an average).  \hfill \\
$H[Y \mid X=x] = -\sum\limits_y P(Y=y \mid X=x) \cdot \log_2 P(y \mid X=x)$   \hfill \\
 $H[Y \mid X=x] =  E[S(Y \mid X=x)]  $  \hfill \\
Note that we are summing over y because we are specifying x. \hfill \\
\hfill \\

For a particular value of X:  \hfill \\
$H[Y \mid X] = \sum\limits_x p(x) H[Y \mid X=x]$  \hfill \\
\hfill \\

Back to table above: 
\begin{align*}
	H[Y \mid X=0] &= ?  \\
	& \mbox{look only at X=0 in table.}  \\
	&= -[0 + 1 \log_2]  \\
\end{align*}
Now that you know X=0, entropy goes to 0.   \hfill \\  \hfill \\

$H[Y \mid X=1] = 1$: You know \textit{less} if you know X=1. \hfill \\ \hfill \\

Now use 
$H[Y \mid X] = \frac{1}{3}(0) + \frac{2}{3}(1) = 2/3$ \hfill \\
Given X, you know more.  Average our the more certain case and the less certain case.  \hfill \\  \hfill \\

Note:  $H[Y \mid X] \leq H[Y]$: knowing something can't make you know less. 

\subsection{Entropy and Information Gain}
\textbf{Information Gain} -  $IG(X) = H(Y) - H(Y \mid X)$ \hfill \\
Y is the node on top.  X are the nodes below.  He might have used lower case.  \hfill \\
\textbf{Class Example}: If $X_1$ is a node for a split, and you want to know the information gain for that node, you:
\begin{itemize}
	\item calculate entropy of the split.  Find Entropy of each branch of the split, and the fraction of points that were channeled to each split.  E.g. %http://courses.cs.washington.edu/courses/cse446/16wi/Slides/2_DecisionTrees_Part2.pdf
	\begin{align*}
		\{T, T, T, T, T, F\} & \rightarrow \{T, T, T, T\} \mbox{ (for } X_1=T), \{T, F\} \mbox{ (for } X_1=F) \\
				& \rightarrow P(X_1 = T) = 4/6, P(X_1=F) = 2/6 \\
				& \rightarrow H(X_1 = T) =  (1*\log_2 1 + 0*\log_2 0) = 0 \\
				& \rightarrow H(X_1 = F) =  \frac{2}{6} (\frac{1}{2} \log_2 \frac{1}{2} + \frac{1}{2} \log_2 \frac{1}{2})  \\
				& = 1 \mbox{ (uniform distribution)} \\
				H(Y|X_1)  =  - \frac{4}{6} & (1*\log_2 1 + 0*\log_2 0) - \frac{2}{6} (\frac{1}{2} \log_2 \frac{1}{2} + \frac{1}{2} \log_2 \frac{1}{2}) \\
							& = 2/6
	\end{align*}
	\item find the entropy of the unsplit data:  $\{T, T, T, T, T, F\} \rightarrow -(5/6) \log_2(5/6) - (1/6) \log_2(1/6) = 0.65$  % check w/ google: -(5/6)*log_2(5/6) - (1/6)*log_2(1/6) == 0.65. 
	\item subtract the weighted average of the split entropies from the original: $IG(X_1) = H(Y) - H(Y|X_1) = 0.65 - 0.33$
\end{itemize}


%\textbf{Conditional Information Gain}:    \hfill \\ %$H(y|x) = -\sum P(y|x) log_2$ 
  \hfill \\

\begin{itemize}
	\item Low uncertainty $\leftrightarrow$ Low entropy.
	\item Lowering entropy $\leftrightarrow$ More information gain. 
\end{itemize}


\subsection{Bits}
If you use log base 2 for entropy, the resulting units are called bits (short for binary digits). \hfill \\ % book pg 57
How many things can you encode in 15 bits? $2^{25}$.  \hfill \\  % 1/11/2015 Lecture


\subsection{Common notation}
\textbf{semicolon versus $|$ in probabilities}: \hfill \\
E.g. $P(X ; \theta)$ vs $P(X | \theta)$

$|$ is for random variables and $;$ is for parameters.  

Andrew Ng verbalizes the semicolon as "parameterized by."  
So $f(x ; \theta)$ would be spoken as "f of x parameterized by theta"
	






\section{Decision Trees}
\smallskip \hrule height 2pt \smallskip

Summary:  \hfill \\
\begin{itemize}
	\item One of the most popular ML tools.  Easy to understand, implement, and use.  Computationally cheap (to solve heurisRcally). 
	\item Uses informaton gain to select attributes (ID3, C4.5,�) 
	\item Presented for classification, but can be used for regression and density estimation too 
	\item Decision trees will overfit!!! 
	\item Must use tricks to find �simple trees�, e.g.,  (a) Fixed depth/Early stopping, (b) Pruning, (c) Hypothesis testing
	\item Tree-based methods partition the feature space into a set of rectangles.  % Elem of Stat. Learning pg 305
	\item  Interpretability is a key advantage of the recursive binary tree.
\end{itemize}

Pros: 
\begin{itemize}
	\item easy to explain to people
	\item more closely mirror human decision-making than do the regression and classification approaches
	\item can be displayed graphically, and are easily interpreted even by a non-expert
	\item can easily handle qualitative predictors without the need to create dummy variables
\end{itemize} 

Cons: 
\begin{itemize}
	\item  trees generally do not have the same level of predictive accuracy as some of the other regression and classification approaches
	\item can be very non-robust. A small change in the data can cause a large change in the final estimated tree
\end{itemize} 

Vocab:
\begin{itemize}
	\item \textbf{classification tree} - used to predict a qualitative response rather than a quantitative one %ISL pg 311
	\item \textbf{regression tree} - predicts a quantitative (continuous) variable. 
	\item \textbf{depth of tree} -the maximum number of queries that can happen before a leaf is reached and a result obtained %wikipedia
	\item \textbf{split} - 
	\item \textbf{node} - synonymous with split.  A place where you split the data. 
	\item \textbf{node purity} - 
	\item \textbf{univariate split} -  A split is called univariate if it uses only a single variable, otherwise multivariate.
	\item \textbf{multivariate decision tree} - can split on things like A + B or Petal.Width / Petal.Length < 1.  If the multivariate split is a conjunction of univariate splits (e.g. A and B), you probably want to put that in the tree structure instead. % http://www.ismll.uni-hildesheim.de/lehre/ml-07w/skript/ml-2up-04-decisiontrees.pdf
	\item \textbf{univariate decision tree} - a tree with all univariate splits/nodes.  E.g. only split on one attribute at a time. 
		% http://www.ismll.uni-hildesheim.de/lehre/ml-07w/skript/ml-2up-04-decisiontrees.pdf
	\item \textbf{binary decision tree} 
	\item \textbf{argmax} - the input that leads to the maximum output
	\item \textbf{greedy} - at each step of the tree-building process, the best split is made at that particular step, rather than looking ahead and picking a split that will lead to a better tree in some future step. % An Introduction to Statistical Learning.pdf pdf pg 320
	\item \textbf{threshold splits} -   % lec http://courses.cs.washington.edu/courses/cse446/16wi/Slides/2_DecisionTrees_Part2.pdf
	\item \textbf{random forest}: an ensemble of decision trees which will output a prediction value. Each decision tree is constructed by using a random subset of the training data.  % https://www.kaggle.com/c/titanic/details/getting-started-with-random-forests
\end{itemize}



Protocol:
\begin{enumerate}
	\item Start from empty decision	tree
	\item Split on next best attribute (feature).   
		\begin{itemize}
			\item Use something like information gain to select next attribute. $\displaystyle \argmax_i IG(X_i) = \argmax_i H(Y) - H(Y | X_i) $ %\( \displaystyle \argmin_x \)
		\end{itemize}	
	\item Recurse
\end{enumerate}

When do we stop decision trees?
\begin{itemize}
	\item Don�t split a node if all matching records have the same output value
	\item Only split if all of your bins will have data in them.  His words: "none of the attributes can create multiple nonempty  children." He also said "no attributes can distinguish", and showed that for the remaining training data, each category only had data for one label.  And third, "If all records have exactly the  same set of input attributes then don�t recurse"
\end{itemize}
He noted that you might not want to stop splitting just because all of your information nodes have zero information gain. 
You would miss out on things like XOR.  %http://courses.cs.washington.edu/courses/cse446/16wi/Slides/2_DecisionTrees_Part2.pdf

Decision trees will overfit.  If your labels have no noise, the training set error is always zero. 
To prevent overfitting, we must introduce some bias towards simpler trees.
Methods available: 
\begin{itemize}
	\item Many strategies for picking simpler trees 
	\item Fixed depth 
	\item Fixed number of leaves 
	\item Or something smarter� 
\end{itemize}

One definition of \underline{overfitting}:  If your data is generated from a distribution $D(X,Y)$ and you have a hypothesis space $H$: \hfill \\
Define errors for hypothesis $h \in H$: training error = $error_{train}(h)$,  Data (true) error = $error_D(h)$.  
The hypothesis $h$ overfits the training data if there exists an $h'$ such that $error_{train}(h) < error_{train}(h')$ and $error_{D}(h) > error_{train}(D)$.
In plain english, if there is an alternative hypothesis that gives you more error on the training data but less error in the test data then you have overfit your data.  
\hfill \\ \hfill \\

\underline{How to Build Small Trees}  \hfill \\
Two reasonable approaches: 
\begin{itemize}
	\item Optimize on the held-out (development) set.  If growing the tree larger hurts performance, then stop growing.  But this requires a larger amount of data
	\item Use statistical significance testing.  Test if the improvement for any split it likely due to noise.  If so, don�t do the split.  Chi Square test w/ MaxPchance = something like 0.05. 
\end{itemize}

\underline{Pruning Trees}  \hfill \\
Start at the bottom, not the top.  The top is most likely to have your best splits.  
In this way, you only cut high branches if all the branches below were cut. 

Don't use the validation set for pruning.   % TA canvas note 1/24/2015. 
\textbf{Your code should never use the validation set.}
The validation set is for \textbf{you} to learn from; the code will always learn from the training set.


\underline{Classification vs. Regression Trees}  \hfill \\
In class we mostly discussed nodes with categorical attributes.  
You can have continuous attributes (see HW1).  
You can also have either discrete or continuous output.  
When output is discrete, you can chose your splits based on entropy.
If it is continuous, you need to do something more like least squares.  
For regression trees, see pg 306 from \href{http://www-bcf.usc.edu/~gareth/ISL/}{ISL} or pg 307 of \href{http://web.stanford.edu/~hastie/local.ftp/Springer/OLD/ESLII_print4.pdf}{ESLII}.

\underline{For discrete data}: \hfill \\
"For discrete data, you can't split twice on the same feature. Once you've moved down a branch, you know that all data in that branch has the same value for the splitting feature."  % TA board 1/24/2015

\underline{For continuous data}: \hfill \\
More computationally expensive than discrete data. 
Often can try to change continuos data to categorical. 
Might lose some smoothness for real numbers, but might be worth it 

\underline{K-fold validation versus using a held-out data set}: \hfill \\
If you have enough data to pull out a held-out set, that is preferable to K-fold validation.  % Farhadi said to me 1/25/2016. 
% This says the opposite: http://stats.stackexchange.com/questions/104713/hold-out-validation-vs-k-fold-validation


\section{Maximum Likelihood \& Maximum a Posteriori}
\smallskip \hrule height 2pt \smallskip
(Also see paragraph at the end of this PDF near vocab for a MLE/MAP comparison.) \hfill \\

\underline{Vocab}
\begin{itemize}
	\item\textbf{likelihood}: the probability of the data given a parameter.  E.g. $P(D | \theta)$ (for discrete like Binomial).  
	Need not a pdf; need not be normalized.   %  http://www.robots.ox.ac.uk/~az/lectures/est/lect34.pdf  + Erick
 	\item \textbf{log-likelihood}: lower-case: $l(\theta|x) = \log L(\theta | x)$
	\item \textbf{maximum likelihood} (ML): 
	\item \textbf{MLE}: Maximum Likelihood Estimation. 
	\item \textbf{PAC}: Probability Approximately Correct. 
	\item \textbf{Posterior}: the likelihood times the prior, normalized  % Murphy 2012 pg 70
\end{itemize}
 
\underline{MLE}: Maximum Likelihood Estimation \hfill \\
Choose $\theta$ to maximize probability of D. \hfill \\
Set derivative of \_\_ to zero and solve.  If function is multivariate, set each partial derivative to zero and solve. \hfill \\
$\hat{\theta} = \argmax_\theta P(D | \theta) = \argmax_\theta \ln P(D | \theta) $ \hfill \\
Note we are using $\ln$, not $\log_2$ as we did for entropy above.  Want it to cancel exponents now. 
 \hfill \\
 
\hfill \\
\underline{Binomial Distribution} \hfill \\
Assumes i.i.d: $D=\{x_i | i=1 \dots n\}, P(D | \theta) = \prod_i P(x_i \mid \theta)$. \hfill \\
Likelihood function: $P(D | \theta) = \theta^{\alpha_H} (1-\theta)^{\alpha_T}$  \hfill \\
P(heads) = $\theta$, P(tails) = $1 - \theta$ \hfill \\
\begin{align*} 
	\hat{\theta} &= \argmax_\theta \ln P(D| \theta) \\
	 	 &= \argmax_\theta \ln \theta*{\alpha H}(1-\theta)^{\alpha T}
\end{align*}
Find optimal theta by setting the derivative to zero: 
\begin{align*} 
	\frac{d}{d\theta} \ln P(D| \theta) &= \frac{d}{d\theta}  \ln \theta*{\alpha H}(1-\theta)^{\alpha T} \\
	 	 &= \argmax_\theta \ln \theta*{\alpha H}(1-\theta)^{\alpha T} \\
		 & = \dots = \frac{\alpha_H}{\alpha_H  + \alpha_T}
\end{align*}

For Binomial, there is exponential decay in uncertainty with \# of observations.  % slide 7 at http://courses.cs.washington.edu/courses/cse446/16wi/Slides/3_PointEstimation.pdf
You can also find the probability that you are approximately correct (see \href{http://courses.cs.washington.edu/courses/cse446/16wi/Slides/3_PointEstimation.pdf}{notes}).  \hfill \\
$P(|\widehat{\theta} = \theta*| \geq \epsilon) \leq 2e^{-2N\epsilon^2}$.  Can calculate N (\# of flips) to have error less than $\epsilon$ with probability of being incorrect $\delta$.  Your sensitivity depends on your problem; error on stock market data might cost billions. 

What if you had prior beliefs?  Use MAP instead of MLE.



\section{Bayesian Learning}
\smallskip \hrule height 2pt \smallskip

Rather than estimating a single $\theta$, we obtain a  distribution over possible values of $\theta$.

For small sample size, prior is important! 

Use Bayes' Rule:
$ \displaystyle P(\theta | D) = \frac{P(D | \theta) P(\theta)}{P(D)}$
\begin{itemize}
	\item \textbf{Posterior}: $P(\theta | D)$
	\item \textbf{Data Likelihood}: $P(D | \theta) $
	\item \textbf{Prior}: $P(\theta)$
	\item \textbf{normalization}: $P(D)$
\end{itemize}
Or equivalently, $P(\theta | D) \propto P(D | \theta) P(\theta)$

If you have a uniform prior, you just do MLE.  \hfill \\
$P(\theta) \propto 1 \rightarrow P(\theta | D) \propto P(D | \theta)$

\underline{Vocab}
\begin{itemize}
	\item \textbf{prior}: 
	\item \textbf{prior distribution}: 
	\item \textbf{posterior}: 
	\item \textbf{posterior distribution}: 
	\item \textbf{MAP}:
\end{itemize}

\hfill \\
\underline{Thumbtack Problem}
\begin{itemize}
	\item use Binomial likelihood:  $P(D | \theta) = \theta^{\alpha_H} (1-\theta)^{\alpha_T}$
	\item To get a simple posterior form, use a conjugate prior.  Conjugate prior of Binomial is the Beta Distribution.  See \href{http://courses.cs.washington.edu/courses/cse446/16wi/Slides/3_PointEstimation.pdf}{slides} for math. 
	\item The Beta prior is equivalent to extra thumbtack flips.  As $N \rightarrow \infty$, the prior is �forgotten�.  But for small sample size, prior is important.  
\end{itemize}

If you are measuring a continuous variable, Gaussians are your friend. 

\section{Gaussians}
\smallskip \hrule height 2pt \smallskip

Properties of Gaussians: 
\begin{itemize} 
	\item Affine transformation (multiplying by a scalar and adding a constant) are Gaussian.
		If X $\sim$ N($\mu$,$\sigma^2$) and Y = aX + b, then Y $\sim$ N($a\mu+b, a^2\sigma^2$) 
  	\item Sum of Gaussians is Gaussian.  
			If X $\sim$ N($\mu_X, \sigma^2_X$), 
			Y $\sim$ N($\mu_Y, \sigma^2_Y$), 
			and Z = X+Y, then 
			Z $\sim$ N($\mu_X+\mu_Y, \sigma_X^2 +\sigma_Y^2$)
	\item  Easy to differentiate.
\end{itemize}

Learn a Gaussian: $P(x | \mu, \sigma) = \frac{1}{\sigma \sqrt{2 \pi}}e^\frac{-(x-\mu)^2}{2\sigma^2}$. \hfill \\
MLE for Gaussian: Prob of i.i.d. samples D = $\{x_1, \dots, x_N\}$:  \hfill \\
$\displaystyle  P(D|\mu, \sigma) = ( \frac{1}{\sigma \sqrt{2 \pi}})^N \prod_{i=1}^N e^\frac{-(x_i-\mu)^2}{2\sigma^2}$.   \hfill \\
Note: it is \underline{not} $P(\mu, \sigma | D)$, like I thought in class.  \hfill \\
Find $\mu_{MLE}$, $\sigma_{MLE} = \argmax_{\mu, \sigma} P(D | \mu, \sigma)$.  \hfill \\

Log-likelihood:  $ \displaystyle \ln P(D | \mu, \sigma) = \ln[\mbox{thing above}] = -N \ln \sigma \sqrt{2\pi} - \sum_{i=1}^N \frac{(x_i - \mu)^2}{2\sigma^2}$.  \hfill \\
Differentiate w.r.t. $\mu$ and set = 0.  End up with $ \displaystyle \widehat{\mu} = \frac{1}{N} \sum_{i=1}^N x_i$.  \hfill \\
Differentiate w.r.t. $\sigma$ and set = 0.  End up with $ \displaystyle \widehat{\sigma}^2_{MLE} = \frac{1}{N} \sum_{i=1}^N (x_i-\widehat{\mu})^2$.  \hfill \\
But actually, that leads to a biased estimate, so people actually use  $ \displaystyle \widehat{\sigma}^2_{unbiased} = \frac{1}{N-1} \sum_{i=1}^N (x_i-\widehat{\mu})^2$  \hfill \\

The conjugate priors: mean: use Gaussian prior:  $ \displaystyle  P(\mu | \nu, \lambda) = \frac{1}{\lambda \sqrt{2 \pi}}e^\frac{-(\mu - \nu)^2}{2\sigma^2} $.  (Instead of $\sigma$, use $\lambda$ and replace the $(x-\mu)^2$ with $(\mu - \nu)^2$).  \hfill \\
For variance: use Wishard Distribution:  

\section{Linear Regression}
\smallskip \hrule height 2pt \smallskip

\underline{Ordinary Least Squares} \hfill \\

Notation:
\begin{itemize}
	\item \textbf{$x_i$}: an input data point.  \_\_ rows by \_\_ columns. 
	\item \textbf{$y_i$}: a predicted output
	\item \textbf{$\widehat{y_i}$}: a predicted output
	\item \textbf{$\widehat{y}$}: 
	\item \textbf{$w_k$}: weight k
	\item \textbf{$\bm{w}*$}:
	\item \textbf{$f_k(x_i)$}
	\item \textbf{$t_j$}: the output variable that you either have data for or are predicting. 
	\item \textbf{$t(\bm{x})$}: Data.  "Mapping from x to t(x)"
	\item \textbf{$H$}: $H = \{ h_1, \dots, h_K \}$.  Basis functions.  In the simplest case, they can just be the value of an input variable/feature or a constant (for bias).  
\end{itemize}

\underline{Vocab}:
\begin{itemize}
	\item \textbf{basis function}
	\item \textbf{bias} - like the intercept in a linear equation.  The part that doesn't depend on the features. 
	\item \textbf{hyperplane} - a plane, usually with more than 2 dimensions. 
	\item \textbf{input variable} - a.k.a. feature.  % https://en.wikipedia.org/wiki/Dependent_and_independent_variables
		E.g. a column like CEO salary for rows of data corresponding to different companies.
	\item \textbf{response variable} - synonyms: "dependent variable", "regressand", "predicted variable", "measured variable", "explained variable", "experimental variable", "responding variable", "outcome variable", and "output variable".   E.g. a predicted stock price.   
	\item \textbf{regularization} -  introducing additional information in order to solve an ill-posed problem or to prevent overfitting. 
	% https://en.wikipedia.org/wiki/Regularization_(mathematics)
	E.g. applying a penalty for large parameters in the model. 
	\item \textbf{ridge regression} - 
\end{itemize}

\underline{Ordinary Least Squares}: \hfill \\
total error = $\displaystyle \sum_i (y_i-\hat{y_i})^2 = \sum_i(y_i - \sum_k w_k f_k(x_i))^2$ \hfill \\
Under the additional assumption that the errors be normally distributed, OLS is the maximum likelihood estimator. \hfill \\ % https://en.wikipedia.org/wiki/Ordinary_least_squares
?? Use words to describe what subset of regression in general this is.  What is ordinary? What are we limiting?  \hfill \\
 \hfill \\

The regression problem: \hfill \\
Given basis functions $\{ h_1, \dots, h_K \}$  with $h_i(\bf{x}) \in \mathbb{R}$,  \hfill \\
	find coefficients $\bm{w} = \{ w_1, \dots, w_k \}$.  \hfill \\%  
$t(\bm{x}) \approx \widehat{f}(\bm{x}) = \sum_i w_i h_i(\bm{x})$ 

This is called linear regression b/c it is linear in the parameters. 
We can still fit to nonlinear functions by using nonlinear basis functions. 
Minimize the \textbf{residual squared error}: \hfill \\
$ \displaystyle \bm{w}* = \argmin_{\bm{w}}  \sum_j (t(\bm{x}_j) - \sum_i w_i h_i(\bm{x}_j))^2$
\hfill \\  \hfill \\

For fitting a line in 2D space, your basis functions are $\{ h_1(x) = x, h_2(x) = 1 \}$  \hfill \\  \hfill \\

To fit a parabola, your basis functions could be $\{ h_1(x) = x^2, h_2(x)=x, h_3(x)=1 \}$.   \hfill \\
Want a 2D parabola? Use $\{ h_1(x) = x_1^2, h_2(x)=x_2^2, h_3(x)=x_1 x_2, \dots \}$. \hfill \\
Can define any basis functions $h_i(\bm{x})$ for n-dimensional input $\bm{x} = <x_1, \dots, x_n>$
\hfill \\  \hfill \\

\underline{Regression: matrix notation}: \hfill \\
\begin{align*}
	\bm{w}* &= \argmin_w \sum_j(t(\bm{x}_j - \sum_i w_i h_i(\bm{x}_j))^2  \\
	\bm{w}* &= \argmin_w (\bm{Hw} -\bm{t})^T (\bm{Hw} -\bm{t})
\end{align*}
$  (\bm{Hw} -\bm{t})^T (\bm{Hw} -\bm{t})$ is the residual error. 
\includegraphics[width=3in]{figures/Least_squares_matricies.pdf}

\underline{Regression: closed form solution}:  % derivation: http://courses.cs.washington.edu/courses/cse446/16wi/Slides/4_LinearRegression.pdf
\begin{align*}
	\bm{w}* = \argmin_w (\bm{Hw} -\bm{t})^T (\bm{Hw} -\bm{t})  & \\
	\bm{F}(\bm{w}) =  \argmin_w (\bm{Hw} -\bm{t})^T (\bm{Hw} -\bm{t}) & \\
	\triangledown_{\bm{w}}\bm{F}(\bm{w}) = 0 \\
	2 \bm{H}^T (\bm{H}\bm{w}-\bm{t}) = 0  & \\
	(\bm{H}^T\bm{H}\bm{w}) - \bm{H}^T\bm{t} = 0 & \\
	\bm{w}* = (\bm{H}^T\bm{H})^{-1}\bm{H}^T\bm{t} &
\end{align*}

\includegraphics[width=3in]{figures/Regression_matrix_math.pdf}

Linear regression prediction is a linear function plus Gaussian noise:  \hfill \\
$t(\bm{x}) = \sum_i w_i h_i(\bm{x}) + \epsilon $ \hfill \\
We can learn $\bf{w}$ using MLE: 
$P(t | x, w, \sigma) = \frac{1}{\sigma \sqrt{2 \pi}} e^\frac{-[t - \sum_i w_i h_i(x)]^2}{2 \sigma^2}$
Take the log and maximize with respect to w:  (maximizing log-likelihood with respect to w) \hfill \\
$\displaystyle \ln P(D | \bm{w}, \sigma) = \ln(\frac{1}{\sigma \sqrt{2 \pi}})^N \prod_{j=1}^N e^\frac{-[t_j - \sum_i w_i h_i(x_j)]^2}{2 \sigma^2}$ \hfill \\
Now find the w that maximizes this: \hfill \\
$\argmax_w \ln(\frac{1}{\sigma \sqrt{2 \pi}})^N + \sum_{j=1}^N \frac{-[t_j - \sum_i w_i h_i(x_j)]^2}{2 \sigma^2}$ \hfill \\
the first term isn't impacted by $w$ so  \hfill \\
$= \argmax_w  \sum_{j=1}^N \frac{-[t_j - \sum_i w_i h_i(x_j)]^2}{2 \sigma^2}$ \hfill \\
switch to $\argmin_w$ when we divide by -1.  The numerator is constant.:  \hfill \\
$= \argmin_w  [t_j - \sum_i w_i h_i(x_j)]^2 $ \hfill \\

\textbf{Least-squares Linear Regression is MLE for Gaussians!!!}  \hfill \\ \hfill \\

\underline{Regularization in Linear Regression}  \hfill \\


 

\section{Naive Bayes}
\smallskip \hrule height 2pt \smallskip
Parameters are from data statistics; probabilistic interpretation.  \hfill \\ % lec 7 end
Train on Y conditioned on x. \hfill \\  % Friday wk 4 audio transcription.
Bayes: $P(Y=X) \propto P(X|Y)P(Y)$.  
But that is ($2^N$) parameters, and you might not be able to make a model with that many ($N$) parameters if your training data has a lot of features.
So you can add some assumptions that take you from $2^N$ parameters down to $2N$ of them. \hfill \\  % Friday wk 4 audio transcription.

Study the problem feature by feature.  Assume chunks of features are conditionally independent.  This is a false assumption, but often works.  
(Features are probably not independent, but it turns out to be reasonably safe to assume so.)
MAP is the foundation for Naive Bayes classifiers.  % http://www.cs.cmu.edu/~tom/10601_sp08/slides/recitation-mle-nb.pdf
 \hfill \\
  \hfill \\
Do we do optimization with NB?  
No.  We start with the joint distribution: $P(x|y)*prior$, $\dots$, take the max.  % week 6 audio
    Not ML optimizing.  Not required to optimize every time we learn something. % week 6 audio

Vocabulary
\begin{itemize}
	\item \textbf{$h_{NB}(x)$} the function that returns the best class.  %Erick 2/4/2016
	\item Loss function:
	in Naive Bayes the loss function is the negative log likelihood of the data given the parameters: $- \ln(P(X, y | w))$.  % TA e-mail 3/15/2016
\end{itemize}

Advantages:
\begin{itemize}
	\item Fast to train (single scan/pass through data). Fast to classify  % http://www.cs.ucr.edu/~eamonn/CE/Bayesian%20Classification%20withInsect_examples.pdf
	\item Not sensitive to irrelevant features  % http://www.cs.ucr.edu/~eamonn/CE/Bayesian%20Classification%20withInsect_examples.pdf
	\item Handles real and discrete data   % http://www.cs.ucr.edu/~eamonn/CE/Bayesian%20Classification%20withInsect_examples.pdf
	\item Handles streaming data well   % http://www.cs.ucr.edu/~eamonn/CE/Bayesian%20Classification%20withInsect_examples.pdf
	\item conditional independence assumption $\rightarrow$ we don't need to see occurrences of joint assignments to estimate their probabilities.  % HW 2 forum
\end{itemize}
Disadvantages:
\begin{itemize}
	\item Assumes independence of features  % http://www.cs.ucr.edu/~eamonn/CE/Bayesian%20Classification%20withInsect_examples.pdf
	\item All observations have equal weight in prediction. % http://www.cs.ucr.edu/~eamonn/CE/Bayesian%20Classification%20withInsect_examples.pdf
\end{itemize}

Vocab:
\begin{itemize}
	\item \textbf{prior}:  $P(Y)$, the probability of a label/class.
	\item \textbf{Likelihood}: $P(\bm{X} | Y)$. 
	%\item 
	%\item \textbf{}
\end{itemize}
 
\subsubsection{Conditional independence}
$X$ is conditionally independent of $Y$ given $Z$ if the probability distribution
for $X$ is independent of the value of $Y$, given the value of $Z$: \hfill \\
$(\forall i, j, k)$ $P(X=i | Y = j | Z=k) = P(X=i | Z=k)$.  \hfill \\
E.g. P( Thunder $|$ Rain, Lightening) = P(Thunder $|$ Lightning) \hfill \\
Equivalent to P(X, Y $|$ Z) = P(X $|$ Z) P(Y $|$ Z)  \hfill \\ \hfill \\

TODO: put in plain english. 

HW 2 forum: \hfill 
Naive Bayes assumption (is): \hfill \\
$P(D1, D2|H) = P(D1|H)P(D2|H)$, everything else follows from the rules of probability.


\subsubsection{Naive Bayes Assumption}
Features are independent given the class: \hfill \\
\begin{align*}
	P(X_1, X_2 | Y) &= P(X_1 | X_2, Y) P(X_2 | Y)  \\
				&= P(X_1 | Y) P(X_2| Y)      \\
				& \mbox{more generally:}      \\
	P(X_1, \dots, X_n | Y) &=  \prod_i P(X_i | Y)			
\end{align*}

This reduces the number of parameters a lot!
Say you had 5 features.  
Before this assumption, each of your 5 features could be dependent.   Then you have to assign a probability to each state.  If each is binary, then you can say $2^5$.  
After this assumption, you would just have 5 parameters.   \hfill\\ \hfill \\

\subsubsection{Homework clarifications}
\underline{Homework 2 TA notes:}  \hfill \\
Naive Bayes gives us a way to compute the whole joint distribution P(Cold, Headache, Cough, SoreThroat). Once we have this, everything else follows from the sum/chain rules of probability:

From the definition of conditional probability (which is the chain rule in disguise):
P(Cold | notHeadache, Cough, SoreThroat) = P(Cold, not Headache, Cough, SoreThroat)/P(not Headache, Cough, Sore Throat)

From the sum rule we see that the denominator is:
P(not Headache, Cough, SoreThroat) = P(Cold, not Headache, Cough, SoreThroat) + P(not Cold, not Headache, Cough, SoreThroat)

Thus the entire conditional probability can be computed using the joint distribution (which can be computed from the factorization and the estimates from the dataset).

The reason we want to include the denominator is to make sure that P(not Headache, Cough, SoreThroat) is non-zero (as asked in another post). It's not zero here, but if it was that would mean we can't really ask about the prediction P(Cold | not Headache, Cough, SoreThroat), since this is only defined when P(not Headache, Cough, SoreThroat) is non-zero. In practice people just assume this isn't an issue, and directly maximize the joint probability.

\hfill \\
The point of Naive Bayes isn't that $P(H| D1, D2)$ is proportional to $P(H, D1, D2)$, as this is true for any distribution (Here Di are the data, H is the hypothesis for some arbitrary problem). The point is that we assume the conditional independence $P(D1, D2| H)$ is $P(D1|H) P(D2|H)$. No the denominator is not used in practice, but that's a simplifying assumption that only works if none of your conditional probabilities are zero (which in turn means all joint probabilities are nonzero). Using the denominator doesn't defeat the point of Naive Bayes, it just ensures that $P(Cold | not Headache, Cough, SoreThroat)$ is well defined as explained above. \hfill \\

You should be using the conditional independence assumption from Naive Bayes to compute the joint probabilities in the numerator/denominator: e.g. $P(H, D1, D2) = P(H)*P(D1|H)*P(D2|H)$, where $P(H)$, $P(D1|H)$, $P(D2|H)$ are estimated from the data (with or without Laplacian smoothing). The conditional independence assumption means that we don't estimate $P(D1 = True, D2 = False | H)$ from the number of times we see $(D1= True, D2=False)$ together, we split $P(D1=True, D2=False|H)$ into $P(D1=True|H)*P(D2=False|H)$ and estimate each conditional independently. This is a huge simplification, since if we have N binary variables, we would need to see $2^N$ combinations to see all of the joint assignments (D1,...DN).
\hfill \\ \hfill \\

[Cold] is not independent of [Headache], [Cough], or [SoreThroat]. Consider: if [Cold] were in fact not related to these variables, we wouldn't be predicting [Cold] from them. Furthermore, [Cough] and [Headache] are not independent, since both are not independent of [Cold]. However, the Naive Bayes assumption is that once you account for that dependence, [Cough] and [Headache] are independent.

Here's a common-sense description of this. If you go around talking to people, you notice that some have headaches, and some have colds. At first, you are confused and wonder why this might be. Is it, perhaps, that coughing all the time is so annoying that it eventually causes people headaches? But then you realize that there is a common illness, the cold, that causes both headaches and coughs. Aha! you say. So you start asking people not only if they have a headache and a cough, but also whether or not they have a cold. You find that among people with colds, having a headache and having a cough seem like unrelated phenomena. And among people without colds, headaches and colds are again unrelated. This is the world that Naive Bayes assumes.

You ask "Would it be the case then that P(Cd, H, C, S) = P(H, C, S | Cd) P(Cd) ?" Yes. That fact is the definition of conditional probability, and it is a basic fact of probabilities. It is true for all variables, in all situations, no matter what assumptions you make. It is true whether Cd and H are independent or not, whether they represent colds or aliens or whatever. You can always use this fact about any variables. \hfill \\
-------------

Naive Bayes is not about ignoring the denominator. This seems to have been a common misunderstanding.

Naive Bayes is an independence assumption; namely, that your input features are independent given the output feature.

This assumption allows you to estimate probabilities like P(X, Y, Z) in terms of other probabilities P(X, Y) and P(X, Z). Since it relates probabilities to each other, it makes it easier to estimate probabilities from samples. For example, if you are trying to detect if a message is spam, you want to compute P(spam | Hello, I, am, Prince, Albert, of, Nigeria, ...), and you may have never seen any points with text (Hello, I, am, Prince, Albert, of, Nigeria, ...) and so would not have any estimate of the probability that that message is spam. With Naive Bayes, you would rewrite this to in terms of the probabilities P(spam | Hello) and P(spam | I) and P(spam | am) and ..., all of which you can compute because you have seen many messages with the words "Hello", or "Prince", or "Nigeria", and know the associated probabilities of spam.

This question asks you to use the Naive Bayes assumption to estimate a probability value given some data, which is exactly what you use Naive Bayes for.

\subsubsection{Naive Bayes Classifier}
Given:
\begin{itemize}
	\item a prior $P(Y)$  
	\item $n$ conditionally independent features $\bm{X}$ given the class $Y$
	\item calculated likelihood for each $X_i$ of the form $P(X_i | Y)$
\end{itemize}
Your decision rule is:   (note $h_{NB}$ is Naive Bayes, not Neg Binom) 
\begin{align*}
	y^* = h_{NB}(x) &= \argmax_y P(y) P(x_1, \dots, x_n | y)   \\
			&=  \argmax_y P(y) \prod_i P(x_i | y)
\end{align*}

\noindent
Although the assumption that the predictor (independent) variables are independent is not always accurate, it does simplify the classification task dramatically, since it allows the class conditional densities $p(x_k | C_j)$ to be calculated separately for each variable, i.e., it reduces a multidimensional task to a number of one-dimensional ones.
In effect, Naive Bayes reduces a high-dimensional density estimation task to a one-dimensional kernel density estimation. Furthermore, the assumption does not seem to greatly affect the posterior probabilities, especially in regions near decision boundaries, thus, leaving the classification task unaffected.
% http://www.statsoft.com/textbook/naive-bayes-classifier

Na�ve Bayes is NOT sensitive to irrelevant features.  % http://www.cs.ucr.edu/~eamonn/CE/Bayesian%20Classification%20withInsect_examples.pdf
However, this assumes that we have good enough estimates of the probabilities, so the more data the better.

\subsubsection{Digit classification example}
Simplify images of digits to pixels, and assign them True or False for whether they are "on".  \hfill \\
Each input maps to a feature in a vector.  E.g. pixel in the 0th for and 0th column is $F_{0,0}$.   \hfill \\
The Naive Bayes model is: \hfill \\
$ \displaystyle P(Y | F_{0,0}, \dots , F_{15,15}) \propto P(Y) \prod_{i,j} P(F_{i,j} | Y)$.
We assume the features are independent given the class.  
We need to learn the distribution of pixels on at each pixel given each number.  \hfill \\
How to calculate the prior, P(Y): \hfill \\
$\displaystyle P(Y) = \frac{count(Y=y)}{\sum_{y'} Count(Y=y'}$ (denominator is summing over all y values) \hfill \\
How to calculate the likelihood:  \hfill \\
$\displaystyle P(X_i =x | Y=y) = \frac{Count(X_i=x, Y=y)}{\sum_{x'}Count(X_i=x', Y=y)}$


\subsubsection{For binary features, use the Beta prior and MAP.} 
Just like likeihood of binomial previously! 
$\displaystyle P(\theta | D) = \frac{\theta^{\beta_H + \alpha_H - 1}(1-theta)^{\beta_T + \alpha_T - 1}}{B(\beta_H + \alpha_H, \beta_T + \alpha_T)} \approx Beta(\beta_H + \alpha_H, \beta_T + \alpha_T) $
Chose $\theta$ using MAP:  \hfill \\
$\displaystyle  \widehat{\theta} = \argmax_\theta P(\theta | D) = \frac{\alpha_H + \beta_H - 1}{\alpha_H + \beta_H + \alpha_T + \beta_T - 2}$.  \hfill \\
Once again, the Beta prior is equivalent to adding extra observations for each feature. \hfill \\
If you don't have a lot of observations, the prior is important.  \hfill \\
And as the number of observations goes to $\infty$, the prior is "forgotten". 

\subsubsection{Multinomials: Laplace Smoothing}
\underline{Laplace's estimate}: \hfill \\
Pretend you saw every outcome $k$ extra times: \hfill \\
$\displaystyle P_{LAP, k}(x) = \frac{c(x) + k}{N + k|X|}$ \hfill \\
$N = $ number of observations.  \hfill \\
$|X| = $ the number of categories you are counting.  \hfill \\
$k$ is the strength of the prior; how much of the prior information you are going to enforce\hfill \\  \hfill \\

Example: 
$\displaystyle P_{LAP, 0}(X) = \langle \frac{2}{3}, \frac{1}{3} \rangle$.  \hfill \\
Set $k=1$.  $|X|$ is 2.  $N$ is 3.  
$\displaystyle P_{LAP, 1}(X) = \langle \frac{3}{5}, \frac{2}{5} \rangle$  \hfill \\
$\displaystyle P_{LAP, 100}(X) = \langle \frac{102}{203}, \frac{101}{203} \rangle$  \hfill \\

\underline{Laplace for conditionals:}
Smooth each condition independently:  \hfill \\
$\displaystyle  P_{LAP, k}(x|y) = \frac{c(x,y) + k}{c(y) + k |X|}$

\subsubsection{Subtleties of the NB classifier}
\textbf{(1) Usually the features are not conditionally independent}: \hfill \\
$P(X_1, \dots, X_n | Y) \neq \prod_i P(X_i | Y)$ \hfill \\
The actual probabilities $P(Y | \bm{X})$ are often biased towards 0 or 1.  
Nonetheless, NB is the single most used classifier out there.  
It performs well even when the independence assumption is violated.  \hfill \\
\textbf{(2) Overfitting} \hfill \\
Conditional probabilities can easily be calculated as zero. 
Zero probabilities kill that class' chance at being called. \hfill \\
??? If the feature is binary, we can use MAP with a beta prior. 
??? That's equivalent to adding extra observations for each feature. 

\subsubsection{NB for text classification}
\begin{itemize}
	\item Need a feature vector with a suitably small number of features.
		Bag of words model is commonly used. 
	\item $i$ is the $i^{th}$ word
	\item NB assumption(\_\_\_) helps a lot.  
		$P(X_i=x_i |Y=y)$ is just the probability of observing word $x_i$ in a document on topic $y$.  
		$ \displaystyle  h_{NB}(x) = \argmax_y P(y) \prod_{i=1}^{LengthDoc} P(x_i | y)$  
	\item Additional assumption: bag of words model. \hfill \\
		Order of words ignored.  Works really well.  \hfill \\
		$P(X_i = x_i | Y=y) = P(X_k = x_i | Y=y)$  ($k$ is the $k^{th}$ word (?); all positions have the same distribution).  \hfill \\
		$P(y) = \prod_{i=1}^{LengthDoc} P(x_i | y)$
	\item Prior, $P(Y)$, is the fraction of documents of each topic.
	\item Likelihood, $P(X_i | Y)$ is count for how many times you saw the word in documents of this topic.  
		This distribution is shared across all positions $i$.
	\item Testing: Use Naive Bayes decision rule.  \hfill \\
		$ \displaystyle  h_{NB}(x) = \argmax_y P(y) \prod_{i=1}^{LengthDoc} P(x_i | y)$
\end{itemize}

\subsubsection{NB for continuous $X_i$}

\begin{itemize}
	\item $k$ is an index over all possible labels. 
	\item $i$ is the $i^{th}$ feature. Here it is the pixel.
	\item $j$ is the $j^{th}$ training example.  
	\item $X_i^j$ is the $i^{th}$ pixel in the $j^{th}$ training sample. 
	\item $Y^j$ is the label corresponding to the $j^{th}$ training example. 
	\item $y_k$ is the $k^{th}$ label
	\item $j$ is $j^{th}$ training example.  
	\item $\delta(x) = 1$ if x true, else 0. 
	\item $h$: the function that returns the best class.  % Erick 2/4/2016
\end{itemize}


Example: character recognition where the darkness of each pixel is continuous.  \hfill \\
\subsection{Gausian Naive Bayes (GNB) for continuous features}

Find parameter that makes all the data points most likely.
What parameters explain our data best?  \hfill \\
 \hfill \\


Naive Bayes continuous video:  \url{https://www.youtube.com/watch?v=r1in0YNetG8}

	$ \displaystyle P(X_i = x | Y = y_k) = \frac{1}{\sigma_{ik} \sqrt{2 \pi}} e^\frac{-(x- \mu_{ik})^2}{2 \sigma_{ik}^2}$  \hfill \\
\begin{itemize}
	\item $\mu_{ik}$ is the mean of the values for the $i^{th}$ feature for the $k^{th}$ class. \hfill \\
	\item $\sigma_{ik}$ is the standard deviation of the values for the $i^{th}$ feature for the $k^{th}$ class. 
\end{itemize}	
	
Sometimes we assume one or both of these:
\begin{itemize}
	\item variance is independent of $Y$ (i.e. $\sigma_i$)
	\item variance is independent of $X_i$ (i.e. $\sigma_k$)
\end{itemize}
If we assume both, we assume just one $\sigma$ without subscripts. \hfill \\   \hfill \\

Estimating parameters for discrete $Y$ and continuous $X_i$:  \hfill \\
\begin{itemize}
	\item \textbf{mean}:  $\displaystyle \widehat{\mu}_{ik} = \frac{1}{\sum_j \delta(Y^j = y_k)} \sum_j X_i^j \delta(Y^j = y_k)$
		\begin{itemize}
			\item first term: divide by the number of training examples that are of class k. 
			\item second term: summing the continuous input of pixel i for all examples 
				in the training set that match label k.  
			\item so this is just an average brightness for pixel $i$ given class $k$ using 
				all the training data.   
		\end{itemize}
	\item \textbf{variance}:  $\displaystyle  \widehat{\sigma}_{ik}^2 = \frac{1}{\sum_j \delta(Y^j = y_k) - 1} \sum_j (x_i^j - \widehat{\mu}_{ik})^2 \delta(Y^j = y_k)$
		\begin{itemize}
			\item first term: divide by the number of training points of class k minus 1. 
			\item second term: sum the squared difference in brightness of pixel $i$ 
				compared to the mean for that pixel and label. 
		\end{itemize}
\end{itemize}
We don't need to use a Gaussian for the prior, $P(y)$.  We aren't optimizing over it, so it is safe to count.  % audio week 5 

\subsection{When Bayes Classifier is Optimal:}
In Bayes we are learning the function $h$ that produces labels $Y$ based on inputs $\bm{X}$.  
More formally: \hfill \\
$h : \bm{X} \mapsto Y$, or \hfill \\
we are learning "the function h that maps features $\bm{X}$ to labels Y". \hfill \\ \hfill \\

If you know the true $\displaystyle P(Y | \bm{X})$, then \hfill \\
$\displaystyle h_{Bayes} = \argmax_y P(Y=y | \bm{X} = x)$.   \hfill \\  \hfill \\
Note the subscript is Bayes, not Naive Bayes; no assumption of conditional independence.
Also, the conditionality is back to likelihood instead of posterior.  \hfill \\  \hfill \\

Theorem: Bayes (not NB) classifier $h_{Bayes}$ is optimal.  \hfill \\
$error_{true}(h_{Bayes}) \leq error_{true}(h)$, $\forall h$ \hfill \\
\hfill \\
\textit{This is theoretical result: we don't know $P(\bm{x})$.  We can't calculate the true Bayes classifier b/c we don't know the distribution of all the data.)}
We also don't know $P(Y | \bm{X})$, the true class' highest probability.  Usually that's hidden; if we knew it we would go home happy.  


Plain english: the predictions you get from Bayes are better than any other function/prediction available. \hfill \\ \hfill \\
Proof:  
\begin{align*}  % changed notation to capital p, X, Y
	P_h(error) = \int_x P_h & (error | \bm{X}) P(\bm{X}) \\  
		& \mbox{           def. of error: } P_h(error| \bm{x}) = \int_y \delta(h(\bm{X}), Y) P(Y| \bm{X}) \\
		& \mbox{               (note, usually we'd sum over the classes, Y)}  \\
			= \int_x \int_y & \delta(h(\bm{X}), Y) P(Y | \bm{X}) P(\bm{X})  \\
			& \mbox{     (the double integral is zero when $P(Y | \bm{X})$ is largest,} \\
			& \mbox{     which is when the correct classification was selected.)}
\end{align*} 
We are averaging over novel data sets that are generated under the same conditions.

%Note that this is different notation than the rest.  Large vs small P.  
Different notation: delta has a comma in the parentheses and not an equality.   \hfill \\
\begin{itemize}
	\item $P_h(error)$ is probability of error across all classifications. 
	\item $\delta(h(\bm{X}, Y)$ is 1 if your X is classified right and 0 if not. 
	\item $P_h(error| \bm{x})$ is the probability that your classification is wrong.  
	\item $\int_x P_h (error | \bm{X}) P(\bm{X}) $ is the expectation of the errors.  
	\item $\delta(h(\bm{X}), Y)$ is 
\end{itemize}

Proof in words:  ??? 

Aside: note that for one classification y is not a vector.  It is a point. 


\section{Logistic Regression}
\smallskip \hrule height 2pt \smallskip

Another probabilistic approach to classification (categorical predictions).   \hfill \\
Can use discrete or continuous outputs. \hfill \\ % https://www.youtube.com/watch?v=zAULhNrnuL4
\hfill \\ 

\underline{Summary from non-class sources:} \hfill \\
% https://www.youtube.com/watch?v=-Z2a_mzl9LM
We are still using linear regression in the inputs, but putting the result into a sigmoid function. \hfill \\
Recall $w_0 + w_1 x_1 + w_2 x_2 + w_3 x_3 = w^Tx$ and $x = (1, x_1, x_2, x_3)$.  \hfill \\
$P(death|x) = \sigma(w^Tx)$  %https://www.youtube.com/watch?v=-Z2a_mzl9LM
where $\sigma$, the sigmoid function,  converts your regression output into a sigmoid curve. \hfill \\
$\displaystyle \sigma(a) = \frac{1}{1+ e^{-a}} = \frac{1}{1+ e^{-(w_0 + w_1 x_1 + w_2 x_2 + w_3 x_3)}} = \frac{1}{1+ e^{-(w^Tx)}}$   \hfill \\ % https://www.youtube.com/watch?v=-Z2a_mzl9LM
\hfill \\

% https://www.youtube.com/watch?v=_Po-xZJflPM  :
We can convert this to a linear relationship by "taking the logit". \hfill \\
The logit (log odds) is the inverse of the logistic.  \hfill \\ %  https://en.wikipedia.org/wiki/Logistic_regression
$F(x) = \sigma(a)$ above.  It is the probability that the dependent variable equals a case, given some linear combination of the predictors.  It can range from $- \infty$ to $\infty$   % https://en.wikipedia.org/wiki/Logistic_regression
The logit is $\ln \frac{F(x)}{1-F(x)}$, or equivalently, after exponentiating both sides: \hfill \\
$\frac{F(x)}{1-F(x)} = e^{w^Tx}$  \hfill \\
The logit (i.e., log-odds or natural logarithm of the odds) is equivalent to the linear regression expression.

 


Note used odds ratio: $\frac{p}{1-p}$  \hfill \\
$\logit(\frac{1}{1+ e^{-(w^Tx)}}) = \log(\frac{\frac{1}{1+ e^{-(w^Tx)}}}{1-\frac{1}{1+ e^{-(w^Tx)}}}) $  \hfill \\ % https://www.youtube.com/watch?v=_Po-xZJflPM
We can now proceed with linear regression.  \hfill \\
Note that our predictions are now on the log scale; this impacts interpretation of the coefficients.  \hfill \\  %https://www.youtube.com/watch?v=_Po-xZJflPM 

\hfill \\ \hfill \\   

\underline{Lecture's presentation:} \hfill \\

Notation:  \hfill \\
\begin{itemize}
	\item $y^j$: the $j^{th}$ class  %(?)
	\item $x^j$: the $j^{th}$ training example
\end{itemize}

Once again we don't want to try to estimate $P(X,Y)$; that is challenging due to the size of the distribution. \hfill \\
We could make the Naive Bayes assumption and only need to calculate $P(X_i | Y)$, 
but if we want $P(Y|X)$, why not learn that directly?  You can use logistic regression. 
\includegraphics[width=1.5in]{figures/expo.pdf}     \hfill \\
\hfill \\

Reuse ideas from regression, but let the y-intercept define the probability.  \hfill \\
$P(Y=1|\bm{X, w}) \propto exp(w_0 + \sum_i w_i X_i)$  \hfill \\
With normalization constants:  \hfill \\
$\displaystyle  P(Y=0|\bm{X, w}) \frac{1}{1+ exp(w_0 + \sum_i w_i X_i)} $ \hfill \\
$\displaystyle  P(Y=1|\bm{X, w}) \frac{exp(w_0 + \sum_i w_i X_i)}{1+ exp(w_0 + \sum_i w_i X_i)} $ \hfill \\
Logistic function: \includegraphics[width=1in]{figures/logistic.pdf}     \hfill \\
 \hfill \\
 
Making a decision boundary out of logistic equations:  \hfill \\
Output the $Y$ with the highest $P(Y|X)$.   \hfill \\
If binary Y, output Y=1 if $\displaystyle 1 < \frac{P(Y=1|X)}{P(Y=0|X)}$  \hfill \\
That simplifies to just $1 <exp(w_0 + \sum_i w_i X_i)$ or \hfill \\
$0 <w_0 + \sum_i w_i X_i$   \hfill \\
\includegraphics[width=.8in]{figures/logistic_boundary_linear.pdf}     \hfill \\
\textbf{The decision boundary is a line (or hyperplane), hence we have a linear classifier!} \hfill \\  \hfill \\

For $ \displaystyle P(Y=0 | \bm{X,w}) = \frac{1}{1 + exp(w_o + w_1 x_1)}$:  \hfill \\
\includegraphics[width=2in]{figures/decision_boundary_example.pdf}   \hfill \\
(See notes for more $w_0, w_1$ values plotted.)  \hfill \\
In these plots, Y is the probability that the class is 1.    \hfill \\
The red curve is the sigmoid.  The blue line is the decision boundary.  \hfill \\
The decision boundary is from the equation $0 = w_1X + w_0$.  \hfill \\
% Erick advice: ignore the blue lines entirely.  don't need them to find probability distribution. 

\hfill \\
Larger weights result in a sharper curve.  The bias $w_0$ shifts there the middle of the curve is.   \hfill \\
The red sigmoid defines a probability distribution over $Y$ in \{0,1\} for every possible input X. \hfill \\
\hfill \\
The decision boundary leads to $P(Y=0|X, w) = 0.5$ when you are at the $y=0$ point on the line.   \hfill \\
(E/J words:  when the blue line crosses the x axis, that's when the sigmoid curve is above 1/2, which corresponds to classifying it as a no/0.)  \hfill \\
The slope of the line defines how quickly the probabilities go to 0 or 1 around the decision boundary. 
\hfill \\

2D inputs: \hfill \\
For $ \displaystyle P(Y=0 | \bm{X,w}) = \frac{1}{1 + exp(w_o + w_1 x_1 + w_2 x_2)}$:  \hfill \\
\includegraphics[width=2in]{figures/decision_boundary_example-2D.pdf}   \hfill \\

$P(Y=0 | X, w)$ decreases as $w_0 + \sum_i w_i x_i$ increases. 
Again, if you set the stuff inside the exponential to zero, you get the decision boundary hyperplane.

\subsubsection{Finding the w coefficients}
Generative (Naive Bayes) loss function: 
Now $j$ is a data point with observations indexed over $i$.


\begin{align*}
	\ln P(D | \bm{w}) = \sum_{j=1}^N  &  \ln P(x^j, y^j | \bm{w}) \mbox{   } \mbox{    (the full log-likelihood)}\\
					& \mbox{use Bayes' rule to rewrite conditionally}  \\  % J added this line. 
					= \sum_{j=1}^N  &  \ln [P(y^j | x^j, \bm{w}) P(x^j | \bm{w})] \\ % J added this line. 
				=  \sum_{j=1}^N  & \ln P(y^j | x^j , \bm{w}) + \sum_{j=1}^N \ln P(x^j | \bm{w})
\end{align*}

We decide to ignore the 2nd term because it won't help you get better predictions for that data anyway. 
Or, "From a machine learning perspective, "God gave us the data" and we don't care about the 2nd sum."  % Erick 2/6/2016

Professor Farhadi is calling this first time a discriminative (logistic regression) loss function:  \hfill \\
It is helping you discriminate between different classes.  It's not going to help you model the data. 
This is unlike regression; we don't care about the value it puts out.  We only care about what the resulting class is.  \hfill \\

% Erick. 
This is the difference between statistics and machine learning.  We only care about getting the best $\bm{w}$ for discriminating between classes. 

\textbf{Conditional Data Likelihood:} 
"Conditional" because you are conditioning on what $\bm{X}$ is. 
\begin{align*}
	\ln P(D_Y | D_{\bm{X}}, \bm{w}) = \sum_{j=1}^N \ln P(y^j | \bm{x}^j, \bm{w})
\end{align*}
$D_Y$ = ???  \hfill \\
$D_{\bm{X}}$ = ???   \hfill \\
Doesn't waste effort learning $P(X)$.  Focuses on $P(Y| \bf{X})$, which is all that matters for classification. \hfill \\
Discriminative models cann't compute $P(\bm{x}^j | \bm{w})$!  ??? 
\hfill \\

\subsubsection{Conditional Log Likelihood}
(the binary case only).  \hfill \\
$P(Y=0 | \bm{X}, \bm{w}) = \frac{1}{1 + \exp(w_0 + \sum_i w_i X_i)}$  \hfill \\
$P(Y=1 | \bm{X}, \bm{w}) = \frac{\exp(w_0 + \sum_i w_i X_i)}{1 + \exp(w_0 + \sum_i w_i X_i)}$  \hfill \\

($ l( \bm{w} )$ is conditional data log-likelihood.)  
\begin{align*}
	l( \bm{w})  \equiv  &  \sum_j \ln P(y^j | x^j, \bm{w})  \\
	& \mbox{Since $y^j$ is in \{0, 1\}, sum over the two cases: }   \\
	& \mbox{(the $y^j$ and $(1-y^j)$ act like delta functions)}   \\
	l(\bm{w})  =  & \sum_j y^j \ln P(y^j = 1 | x^j, \bm{w}) +(1 - y^j) \ln P(y^j = 0 | x^j, \bm{w})  \\
	& \mbox{plug in the definition of the likelihoods and do algebra to get:}  \\
	=& \sum_j  y^j (w_0 + \sum_i^n w_i x_i^j)  - \ln(1 + \exp(w_0 + \sum_i^n w_i x_i^j)) 
\end{align*}	

While we can't find a closed-form solution to optimize $l(\bm{w})$,  $l(\bm{w})$ is concave so we can to gradient \underline{as}cent. 

\subsubsection{Gradent ascent to optimize w}
Conditional likelihood for Logistic Regression is convex.  (see above)  \hfill \\
\textbf{Gradient}:  \hfill \\
\begin{align*}
	\nabla_w l(\bm{w}) = [\frac{\partial l(\bm{w})}{\partial w_0}, \dots, \frac{\partial l(\bm{w})}{\partial w_n}]'  \hfill \\
\end{align*}
	(The $'$ at the end is for transpose b/c usually a column vector.)  \hfill \\
\textbf{Update Rule:} \hfill \\
\begin{align*}
	\Delta \bm{w} &= \eta \nabla_{\bm{w}} l(\bm{w})
\end{align*}
$\eta$ is the learning rate.  $\eta > 0$.

Your next weights $(t+1)$ become: 
$w_i^{(t+1)} \leftarrow w_i^{(t)} + \eta \frac{\partial l(\bm{w})}{\partial w_i}$   \hfill \\
\hfill \\
Gradient ascent is the simplest  of optimization approaches.  Note that conjugate gradient ascent is much better (see reading).(?)

Vocab \hfill \\
\begin{itemize}
	\item \textbf{discriminative}:  estimates joint probabilities.  E.g. $p(Data, Zebra)$, $p(Data, No Zebra)$. 
	\item \textbf{generative}:  E.g. $p(Zebra | Data)$, $p(No Zebra | Data)$. 
\end{itemize}



Andrew Ng: \hfill \\
* gives numbers between 0 and 1 (good for classification)   \hfill \\
* called "logistic regression" but it is really for classification.  (don't be confused by "regression") \hfill \\
* "sigmoid function" and "logistic function" are essentially synonomous.  \hfill \\
 


\section{Perceptrons}
\smallskip \hrule height 2pt \smallskip

\begin{itemize}
	\item Error driven, not probabilistic. \hfill \\
		\begin{itemize}
			\item Mistake driven rather than model drive. \hfill \\
			\item Parameters from reactions to mistakes  % end of slide set 7
		\end{itemize}
	\item Parameters are from a discriminative interpretation % end of slide set 7
	\item To train, you go through the data until the held-out accuracy maxes out. % end of slide set 7
	\item Note you can scale your $w$ (weight) vector(s) by any constant because all you care about is sign($w \cdot x$).
		This rescales your gamma by that constant too! \hfill \\
\end{itemize}

\subsection{Properties of Perceptron}
\underline{Separability}: some parameters get the training set perfectly correct. \hfill \\
\underline{Convergence}: if the training is separable, the perceptron will eventually converge. \hfill \\
\underline{Mistake Bound}: the maximum number of mistakes (for the binary case) is related to the 
margin or degree of separability: $mistakes \leq \frac{R^2}{\gamma^2}$. \hfill \\

\subsection{Problems with the Perceptron}
\includegraphics[width=2.8in]{figures/perceptron_problems.pdf}

 \subsection{Linear Classifiers}
 Inputs are feature values.  \hfill \\
 Each feature has a weight.  \hfill \\
 Sum is the activation.  activation$_w(x) = \sum_i w_i x_i = w \cdot x$  \hfill \\
 If the activation is positive, chose output class 1.  \hfill \\
 If the activation is negative, chose output class 2.  \hfill \\
 
 \includegraphics[width=1.5in]{figures/linear_classifier_cartoon.pdf}  \hfill \\
 
 For a binary decision rule:   \hfill \\
 In the space of feature vectors: 
 \begin{itemize}
 	\item examples are points
	\item any weight vector is a hyperplane
	\item one side corresponds to y = +1
	\item the other side corresponds to y = -1
	\item ??? The $w \cdot x = 0$ is the solution to the line.
 \end{itemize}
 
 \includegraphics[width=1.5in]{figures/binary_decision_rule.pdf} \hfill \\
 The black line is the decision boundary.     \hfill \\
 $w$ is a vector normal to the decision boundary, and points towards the + label points. 
 
 \subsubsection{Binary Perceptron Algorithm}
 \begin{itemize}
 	\item start with zero weights: $w=0$
	\item for $ t = 1 \dots T$ (T passes over the data):
		\begin{itemize}
			\item for $i = 1 \dots n$ (each training example)
			\begin{itemize}
				\item  Classify with current weights: \hfill \\
					$ y = sign(w \cdot x^i)$ \hfill \\
					(sign($z$)  is $+1$ if $z > 0$, else -1) \hfill \\
				\item if correct (i.e. $y = y^i$), don't change weights.
				\item if it was wrong, update with $w = w + y^i x^i$
			\end{itemize}
		\end{itemize}
 \end{itemize}
Figure showing "you got it wrong:"  \hfill \\
 \includegraphics[width=1.0in]{figures/binary_perceptron_rule.pdf}  \hfill \\
 The -1 is the $y^i$ in the equation above.   \hfill \\
   \hfill \\
  
 \includegraphics[width=3.2in]{figures/perceptron_chugging_example.pdf}
 
 \underline{$w \cdot x$ and the boundary between positive and negative answers} \hfill \\
If a point has w * x = 1000, a small change in x might change w * x' to 999, or 1001, but it surely won't make w * x a negative value. On the other hand, if w * x = 0.0001, even tiny changes to x might make w * x' negative. And by extension, cases where w * x = 0 where even the tiniest change might make w * x' change sign. So the values where w * x = 0 are those that are on the boundary between positive and negative examples.[1]

So \textbf{the equation w * x = 0 defines the boundary between the positive and negative region}. Now let's unpack that statement. w is a fixed vector, while x is a variable point. So think of w = [w1, w2] as constants, and x = [x1, x2] as variables. The equation w * x = 0 is just another way of writing the equation w1 x1 + w2 x2 = 0. But this is a linear equation in two variables, so it defines a line. You can algebraically solve the equation for x2, and that gives you the "standard form" of the equation of a line, which you can then draw.

[1] The boundary of a region is defined as the set of points where even points very close by can be outside that region.

 
 \subsection{Multiclass Decision Rule}
 If we have more than two classes: 
 \begin{itemize}
 	\item we have a weight vector for \textbf{each} class: $w_y$
	\item we calculate an activation for each class: \hfill \\
		activation$_w(x,y) = w_x \cdot x$
	\item the highest activation wins:  \hfill \\
		$y^* = \argmax_y(activation_w(x,y))$ \hfill \\
		"win the vote" 
 \end{itemize}
 
\includegraphics[width=1.5in]{figures/multiclass_decision_rule_planes.pdf}
 
\includegraphics[width=2.5in]{figures/perceptron_multiclass--win_the_vote.pdf}

 \subsubsection{Binary Perceptron Algorithm}
 \begin{itemize}
 	\item start with zero weights: $w_y=0$
	\item for $ t = 1 \dots T$ (T passes over the data):
		\begin{itemize}
			\item for $i = 1 \dots n$ (each training example)
			\begin{itemize}
				\item  Classify with current weights: \hfill \\
					(no more $ y = sign(w_y \cdot x^i)$) \hfill \\
					instead:  $y= \argmax_y w_y \cdot x^i$ \hfill \\
				\item if correct (i.e. $y = y^i$), don't change weights.
				\item if it was wrong, update two vectors: \hfill \\
					$w_y = w_y -x^i$ \hfill \\
					$w_{y^i} = w_{y^i} -x^i$ \hfill \\
					??? Add or subtract from the one that gave you the argmax? ???
			\end{itemize}
		\end{itemize}
 \end{itemize}
 \includegraphics[width=1.0in]{figures/multi_perceptron_rule.pdf}
 
 \subsection{Linear Separability}
 \subsubsection{binary case}
Recall $ \displaystyle  ||x||_2 = \sqrt{\sum_i x_i^2}$ 
 
 The data is linearly separable with margin $\gamma$ if:   \hfill \\
$\exists .w \forall t . y^t (w \cdot x^t) \geq \gamma > 0$.  \hfill \\
Plain english: the data is linearly separable if there exists a w that has a margin greater than zero for all points $t$.  \hfill \\
Note: for $y^t = 1$, $w \cdot x^t \geq \gamma$ and for $y^t = -1$, $w \cdot x^t \leq -\gamma$.  \hfill \\
Plain english: points having label = 1 have a dot product that is greater than $\gamma$, and points that have label = -1 have a dot product that is more negative than $-\gamma$.  \hfill \\
 \includegraphics[width=1.0in]{figures/lin_sep_margin.pdf}
 
 \subsubsection{maximum number of mistakes for training linearly separable binary data}
Here, assume the data is separable with margin $\gamma$ and the weight vector is a unit vector: \hfill \\
In math notation, this is: $\exists w^*$ such that $||w^*||_2 = 1$ and $\forall t. y^t(w^* \cdot x^t) \geq \gamma$.) \hfill \\
Recall that you can scale your $w$ (weight) vector(s) by any constant because all you care about is sign($w \cdot x$),
but that this scales your $\gamma$.  You are just multiplying the equation above by a constant.  \hfill \\

Also assume some maximum radius R for your data points:  
$\forall t. ||x^t||_2 \leq R$ \hfill \\
Then the number of mistakes (parameter updates) made by the perceptron is bounded by 
$\displaystyle mistakes \leq \frac{R^2}{\gamma^2}$. \hfill \\
For full inductive proof, see slides.  
Strategy: let $w^k$ be the weights after the k-th update (mistake).  
Then $k^2 \gamma^2 \leq ||w^k||_2^2 \leq k R^2$ \hfill \\
Therefore $k \leq \frac{R^2}{\gamma^2}$.  \hfill \\
\textbf{If there is a linear separator, the Perceptron will find it!}
		
\subsection{Logistic Regression \& Perceptron similarities}
\underline{Logistic regression}:  \hfill \\
In vector notation, y is in the set $\{ 0, 1\}$. \hfill \\
$w = w + \nu \sum_j [y^j - P(y^j | x^j, w))] x^j$ \hfill \\

\underline{Perceptron}:  \hfill \\
When y is in $\{ 0, 1\}$: \hfill \\
$w = w + [y^j - sign^0(w \cdot x^j)] x^j$  \hfill \\
Note: sign$^0(z) =  + 1$ if $z > 0$ and 0 otherwise.  \hfill \\
\hfill \\
Differences:   \hfill \\
1. Online vs. batch learning.  Logistic regression is batch, Perceptron is on-line.  \hfill \\
2. Logistic is probabilistic and Perceptron is error-driven. 
 




\section{Large-Margin Classifiers \& Support Vector Machines}
\smallskip \hrule height 2pt \smallskip


Vocab: 
\begin{itemize}
	\item \textbf{Large margin classifier}:
	\item \textbf{Support Vector}:
	\item \textbf{Support Vector Machine}:
\end{itemize}

Notation:
\begin{itemize}
	\item $i$ or $j$: the $i^{th}$ or $j^{th}$ data (training) point. 
	\item $\bm{w}$: the weights of the model.  The black line is perpendicular to this vector. 
	\item $\bm{w} \cdot \bm{x}$: the distance from x to the decision boundary. 
	\item $\bm{w} \cdot \bm{x} + w_0$: to move the decision boundary off the origin, you need to shift it by a constant.  
	\item $\bm{w}_0$: a constant that defines a line parallel to the decision boundary and is $w_0$ units away. 
\end{itemize}

\subsection{Intro}
Like perceptron, but maximizes the margin.  

Optimizing a small weight vector: $\displaystyle min_w \: \frac{1}{2}||w||^2$ \hfill  \\  % \: gives 4/18ths of \quad space.
	% https://www.andy-roberts.net/res/writing/latex/hspacing.pdf
and getting points right: $\forall \mbox { points } i, y$  $w_{y^*} \cdot x^i \geq w_y \cdot x^i + 1$
Summary:
\includegraphics[width=2.5in]{figures/svm_overview.pdf}

There are many possible ways to write the same line: \hfill \\
If $\bm{w} \cdot \bm{x} + w_0 = 0$, then these also work: 
$2\bm{w} \cdot \bm{x} + 2w_0 = 0$, $1000\bm{w} \cdot \bm{x} + 1000w_0 = 0$, $\dots$ \hfill \\
Any constant scaling has the same intersection with the z = 0 plane, so you get the same dividing line. \hfill \\
\textbf{This is why we \underline{don't} want} max$_{\gamma, w, w_0}$. \hfill \\  \hfill \\

%Recall that the distance from the the decision boundary to a point is given by $\gamma$
%\includegraphics[width=2.5in]{figures/svm_dist_to_plane.pdf}

The distance from $x^j$ to the decision boundary is given by $\lambda \frac{w}{||w||_2}$ :  \hfill \\
\includegraphics[width=1.5in]{figures/svm_component_norm_to_decision_boundary.pdf}  \hfill \\
$\bar{x}^j$ is the component of $x^j$ that is normal to $w$. \hfill \\
So $x^j = \bar{x}^j + \lambda \frac{w}{||w||_2}$  \hfill \\
(Recall $||w||_2 = \sqrt{\sum_i w_i^2}$)  \hfill \\   \hfill \\

\subsection{motivation for minimizing the norm of the weights}
We can maximize the margin by minimizing $||w||_2$: 
$\gamma = \frac{||w||_2}{w \cdot w} = \frac{1}{||w||_2}$   \hfill \\
Derivation:  \hfill \\
\includegraphics[width=2.7in]{figures/svm_derivation_of_minimizing_weights.pdf} \hfill \\

Intuitive explanation, after 2 key facts:  \hfill \\
\begin{itemize}
	\item Key fact \#1: the bigger your $\bm{w}$ is, the \underline{closer} the line $\bm{w}\bm{x}=1$ is to the line $\bm{w}\bm{x} = 0$.  		Small distance between $\bm{w}\bm{x}=1$ and $\bm{w}\bm{x}=0$ translates to a small margin.  
		Note that you could use any constant in place of 1. 
	\item Key fact \#2: The $w_0$ just shifts the black decision boundary line away from the origin.  
\end{itemize}
Now we can explain why minimizing the norm of the weights leads to the largest margin. 

\begin{itemize}
	\item Imagine $w_0 = 0$, meaning the decision boundary goes through the origin.  \hfill \\
	\item Now let $w = [2, 0]$ for simplicity.    
		Where is the decision boundary?  
		The decision boundary corresponds to $2 x_0 + 0 x_1 = 0$.  
		That means the decision boundary is along $x_0 = 0$, which is a line through the $x_1$ (vertical) axis. 
	\item But the point is that when you set $\bm{w} \cdot \bm{x} = c$ or some other constant.
		Key fact \#1 says that the larger w is, the farther $\bm{w} \cdot \bm{x} = 0$ is from $\bm{w} \cdot \bm{x} = c$  
		In this case where $w_1 = 0$, you have  $w_0 x_0 = c$, or $x_0 = c/w_0 = c/2$.  
		If you want the $=1$ line far from the decision boundary, you need to increase the distance between these lines.
		Since $c$ is fixed, so the only way we can do this is to decrease $w_0$. 
	\item The logic holds true for when $w_1$ is nonzero: you just want to minimize the size of the $\bm{w}$ vector.
		Minimizing the size of $w$ is equivalent to minimizing $||w||_2$ 
\end{itemize}

\includegraphics[width=2.7in]{figures/max_margin_using_canonical_hyperplanes.pdf}

\subsection{SVM recipe}
We want to minimize the norm subject to getting the predictions right:  \hfill \\
$\displaystyle  \min_{w, w_0} \frac{1}{2} ||w||_2^2$ so that $\forall j . y^j(w \cdot x^j + w_0) \geq 1$

We do this with quadratic programming (QP). 
The decision boundary is defined by \textbf{support vectors}, which are data points on the canonical red lines. 
All the points that aren't on the red line are non-support vectors. \hfill \\
 \hfill \\
 If your data is not linearly separable, you can add nonlinear features.
 These are called Kernels, and will be discussed later. 
 
 \subsection{Balancing \# of mistakes and $||w||_2^2$}
 If your data isn't linearly separable, then you might need to allow your $||w||_2$ to be a little bigger and make a few mistakes. 
One option: 
 $\displaystyle  \min_{w, w_0} \frac{1}{2} ||w||_2^2 + M$ (M = \# of mistakes)  \hfill \\
 so that $\forall j . y^j(w \cdot x^j + w_0) \geq 1$
 
 \subsubsection{Hinge Loss}
 One way to balance the number of mistakes and how big $w$ is:  
  $\displaystyle  \min_{w, w_0} \frac{1}{2} ||w||_2^2 + C \sum_j \xi^j$ \hfill \\
 so that $\forall j . y^j(w \cdot x^j + w_0) \geq 1 - \xi^j$ \hfill \\
 C = strength of penalty. \hfill \\
 $\xi$ is size of error for each error.  If the point is classified correctly, $\xi$ is zero.   \hfill \\
 When we want the dot product \& shift $\geq 1 - \xi^j$, the 1 is for the green-line margin, and you are wrong by $\xi$  amount. \hfill \\
 \includegraphics[width=1.5in]{figures/hinge_loss_xis.pdf}
 \hfill \\
 Now we can tune our balance between asking for small weights $\bm{w}$ and asking for perfect classification. 
 \begin{itemize} 
 	\item $C  = \infty \rightarrow$ pressure to separate the data, even if the margin is skinny.
	\item $C = 0 \rightarrow$ fat margins over accuracy. 
\end{itemize}
Use your training set to train C.  \hfill \\ 
Note that if \_\_\_\_\_ $\geq 1$, you don't care if the point is classified wrong.   % had used "margin" but this doesn't make sense!
But if \_\_\_\_\_ $ < 1$, you pay a linear penalty. % had used "margin" but this doesn't make sense! 
\hfill \\

\textbf{Hinge Loss:}  \hfill \\
$\displaystyle \min_{w, w_0} \frac{1}{2} ||w||_2^2 + C \sum_{j=1}^N [1 - y^j(w \cdot x^j + w_0)]$ \hfill \\
$1^{st}$ term is regularization, $2^{nd}$ term is hinge loss.  \hfill \\
Solve by differentiating and set equal to zero.  \hfill \\
There is no closed form solution, but quadratic program is concave.  (??)   \hfill \\
Hinge loss is not differentiable (??), so gradient ascent is a little trickier.  \hfill \\

\subsubsection{Logistic Regression to Minimize Loss}
Logistic regression assumes $P(Y=1 | X=x) = \frac{exp(f(x))}{1 + exp(f(x))}$  \hfill \\
(For Logistic Regression, $f(x)$ was $w_0 + \sum_i w_i X_i$ and we had $Y = \{0, 1\}$)  \hfill \\
Now we have $Y = \{ -1, +1 \}$.  \hfill \\
To maximize data likelihood for $Y = \{ -1, +1 \}$: \hfill \\
$\displaystyle  P(y^i | x^i) = \frac{1}{1 + exp(-y^i f(x^i))}$ \hfill \\
\begin{align*}
	\ln P(D_Y | D_{\bm{X}}, \bm{w}) &= \sum_{j=1}^N \ln P(y^j | x^j, w) \\ 
		& \mbox{plug in the $P$ above} \\
		&= - \sum_{i=1}^N \ln(1 + exp(-y^i f(x^i)))
\end{align*}
Since $-\ln(z) = \ln(1/z)$ we get to minimize this (negative):   \hfill \\
$\displaystyle  \sum_{i=1}^N \ln(1 + exp(-y^i f(x^i))) =  \sum_{i=1}^N \ln(1 + exp(-y^i [w_0 + \sum w_i x_i]))$

\subsubsection{SVMs vs Regularized Logistic Regression}
\textbf{SVM Objective:} \hfill \\
$\displaystyle \argmin_{\bm{w}, w_0} \frac{1}{2} ||w||_2^2 + C \sum_{j=1}^N [1 - y^j f(x^j)]_+$ \hfill \\
where $[x]_+ = max(x , 0)$ \hfill \\
 \hfill \\

\textbf{Logistic Regression Objective:} \hfill \\
$\displaystyle  \argmin_{\bm{w}, w_0} \lambda ||w||_2^2 + \sum_{j=1}^N ln(1 + \exp(-y^j f(x^j)))$ \hfill \\
 \hfill \\
 
 Note that SVM and Logistic Regression have the same $l_2$ regularization term, but different error terms. 

 \includegraphics[width=2.5in]{figures/LR_svm_step_losses.pdf}
 
 \subsection{Multi-class SVMs}
 To do 3 classes, you need to learn 3 classifiers. \hfill \\
 Can't just do $y = \argmax_i w_i \cdot x$ for $i$ classifiers.  
 Wouldn't handle this: 
 \includegraphics[width=0.6in]{figures/multiclass_svm_motivation.pdf}
 \hfill \\  \hfill \\
 
 Instead, we learn 3 classifiers for these 3 symbols: 
 \begin{enumerate}
 	\item + vs $\{ O, - \}$, weights $w_+$
	\item + vs $\{ O, + \}$, weights $w_-$
	\item + vs $\{ +, - \}$, weights $w_O$
 \end{enumerate}
 But to get it working for that set of 3 columns, we need additional constraints.  \hfill \\
 For each class: \hfill \\
 for class $y'$ that is not class $y^j$ ($\forall y' \neq y^j$):  \hfill \\
 And for all classes $j$ ($\forall j$):  \hfill \\
 $w^{y^j} \cdot x^j + w_0^{y^j} \geq w^{y'} \cdot x^j + w_o^{y'} + 1$. \hfill \\
 ($\forall$ = "for all") \hfill \\
 In plain english: ????.   \hfill \\
 ??? Do I have the fact that j is for classes right?  (Could j still be points?)  ??
 \hfill \\
 \hfill \\
 
We can also introduce slack variables as before. 
$\displaystyle \min_{w, w_0} \sum_y ||w^y||_2^2 + C \sum_j \xi^j$ \hfill \\
$w^{y^j} \cdot x^j + w_0^{y^j} \geq w^{y'} \cdot x^j + w_o^{y'} + 1 - \xi^j$. \hfill \\
That's true for class $y'$ that is not class $y^j$ ($\forall y' \neq y^j$), and all classes $j$ ($\forall j$) and for all $\xi^j > 0$ \hfill \\
 \hfill \\
 
 So you can do multiple classes in a one against all approach *or* a multiclass SVM approach.  
 
 \subsection{SVM info from other sources}
 \subsubsection{http://axon.cs.byu.edu/Dan/478/misc/SVM.example.pdf}
 The idea behind SVMs is to make use of a (nonlinear) mapping function that transforms data in input space to data
in feature space in such a way as to render a problem linearly separable.  % http://axon.cs.byu.edu/Dan/478/misc/SVM.example.pdf
The SVM then automatically discovers the optimal separating hyperplane (which, when mapped back into input space, can be a complex decision surface). % http://axon.cs.byu.edu/Dan/478/misc/SVM.example.pdf

 \subsubsection{An Introduction to Statistical Learning}
 \begin{itemize}
 	\item shown to perform well in a variety of settings, and are often considered one of the best �out of the box� classifiers.
	\item The support vector machine is a generalization of a simple and intuitive classifier called the maximal margin classifier. 
		Though the maximal margin classifier is elegant and simple, 
			it cannot be applied to most data sets because it requires that the classes be separable by a linear boundary.
	\item People often loosely refer to the maximal margin classifier, the support vector classifier, 
			and the support vector machine as �support vector machines�. 
	\item Although the maximal margin classifier is often successful, it can also lead to overfitting when p is large.
	\item \textbf{support vector}: data points that �support� the maximal margin hyperplane in the sense that if these points were 				moved slightly then the maximal margin hyper- plane would move as well.   % Intro to statistical learning Ch 9.1
	\item In many cases no separating hyperplane exists, and so there is no maximal margin classifier. 
	\item If you cant separate perfectly, you can settle for almost separating the classes using a so-called soft margin.
	\item Even if a separating hyperplane does exist, then there are instances in which a classifier based on a 
			separating hyperplane might not be desirable. 
	\item The fact that the maximal margin hyperplane is extremely sensitive to a change in a single observation 
			suggests that it may have overfit the training data.
	\item The support vector classifier, sometimes called a \textbf{soft margin classifier}: \hfill \\
		Rather than seeking the largest possible margin so that every observation is not only on the correct side 
			of the hyperplane but also on the correct side of the margin, we instead allow some observations to 
			be on the incorrect side of the margin, or even the incorrect side of the hyperplane. 
		(The margin is soft because it can be violated by some of the training observations.) 
	\item An observation that lies strictly on the correct side of the margin does not affect the support vector classifier.
		Changing the position of that observation would not change the classifier at all, 
			provided that its position remains on the correct side of the margin. 
	\item The \textbf{support vector machine} (SVM) is an extension of the support vector classifier that results from 
		enlarging the feature space in a specific way, using \textbf{kernels}.
	\item A \textbf{kernel} is a function that quantifies the similarity of two observations.
	\item What is the advantage of using a kernel rather than simply enlarging the feature space 
			using functions of the original features? 
		One advantage is computational, and it amounts to the fact that using kernels, 
			one need only compute K(xi, xi? ) for all (n choose 2) distinct pairs i, i?. 
		This can be done without explicitly working in the enlarged feature space.
		This is important because in many applications of SVMs, 
			the enlarged feature space is so large that computations are intractable.
 \end{itemize}
 


 
  





\section{Kernels}
\smallskip \hrule height 2pt \smallskip
If the data is not linearly separable and/or you want a wiggly boundary, use kernels. 

Can give you nonlinear boundaries (good) but the feature space can get really large really quickly. 

Example mapping of data that is not linearly separable to a separable higher dimension space: \hfill \\
\includegraphics[width=2.5in]{figures/example_kernel_separation.pdf}  \hfill \\

General idea: \hfill \\
If $\bm{x}$ is in $R^n$, then $\phi(\bm{x})$ is in $R^m$ for $m>n$. \hfill \\
We can now learn feature weights $\bm{w}$ in $R^m$ and predict using $y = sign(\bm{w} \cdot \phi(\bm{x}))$. \hfill \\
\textbf{A linear function in the higher dimensional space will be non-linear in the original space.}  \hfill \\

\subsubsection{Danger of Mapping to a Higher Dimensional Space:}
The number of terms in a polynomial of degree $d$ for $m$ input features is 
$\displaystyle {d + m - 1}\choose{d} $ $ \displaystyle = \frac{d + m - 1}{d!(m-1)!}$. \hfill \\
This grows fast!  For $d=6$, $m=100$ you get about 1.6 billion terms.  \hfill \\

We are taking a dot product of $m$ rows for the feature space times $d$ columns of polynomial. 
But you also have terms for combinations of features.  E.g. 
\includegraphics[width=0.8in]{figures/example_kernel.pdf}

\subsubsection{Efficient dot-product of polynomials}
For polynomials of degree exactly $d$, the dot product in higher dimensional space can be written as a dot product in lower dimensional space.  \hfill \\
For $m=2$ (2 features): \hfill \\
\includegraphics[width=2.5in]{figures/kernel_dot_polynomials.pdf}  \hfill \\
note: here the $.$ is a dot product, not horizontal space fill like a previous lecture. \hfill \\
$u_1$ is feature 1's value, and $u_2$ is feature 2's value. \hfill \\
Polynomial of exactly degree $d$ means we \underline{don't} have a constant bias term like in homework 3 with $[1, \sqrt{2}u, u^2]$  \hfill \\
This is just factoring.  \hfill \\
\hfill \\

Proof not shown, but for any $d$: \hfill \\
$K(u, v) = \bm{\phi}(u).\bm{\phi}(v) = (u.v)^d$

\subsection{The "Kernel Trick"}
A \textbf{kernel function} defines a dot product in some feature space: 
\begin{align*}
	K(\bm{u}, \bm{v}) =\bm{\phi}(\bm{u}) \cdot \bm{\phi}(\bm{v})
\end{align*}
Where $\bm{u}$, $\bm{v}$ are points in your training or test set, each with their own features. \hfill \\ 
Or $\bm{u}$ could be a data point and $\bm{v}$ could be the $\bm{w}$ of the hyperplane that you need to dot against to classify new points.
 \hfill \\
 
 If $\bm{u}$, $\bm{v}$ each have two dimensions (2 features): \hfill \\

\includegraphics[width=3.1in]{figures/example_kernel_math.pdf} \hfill \\
Thus, a kernel function \textit{implicitly} maps data to a high-dimensional space without the need to compute each $\bm{\phi}(\bm{x})$ explicitly.   \hfill \\
 \hfill \\

Translation for this particular kernel: \hfill \\
We want to get the perks of the high dimensional space given by $\bm{\phi}(\bm{x}) = [1, x_1^2, \sqrt{2}x_1 x_2, x_2^2, \sqrt{2} x_1, \sqrt{2} x_2]$ where $\bm{x}$ is either $\bm{u}$ or $\bm{v}$.  \hfill \\
But this is computationally expensive because there are many terms in that thing.  \hfill \\
If we take the dot product of two vectors $\bm{\phi}(\bm{u})$, $\bm{\phi}(\bm{v})$ we get some magic: \hfill \\
\begin{align*}
	\bm{\phi}(\bm{u}) \cdot \bm{\phi}(\bm{v}) &= [1, u_1^2, \sqrt{2}u_1 u_2, u_2^2, \sqrt{2} u_1, \sqrt{2} u_2]  \cdot \quad \mbox{(line wrapped)} \\
			& \quad \quad [1, v_1^2, \sqrt{2}v_1 v_2, v_2^2, \sqrt{2} v_1, \sqrt{2} v_2]   \\
		&=  (1 + \bm{u} \cdot \bm{v})^2
\end{align*}
Can do $(1 + \bm{u} \cdot \bm{v})^2$ with a dot product that only includes multiplying $[u_1, u_2]^T[v_1, v_2]$.     \hfill \\ 
That's a lot less computation/memory expensive than the full dot product above.  \hfill \\
If that leads to good separation, you are a happy machine learner!   \hfill \\
\textbf{This is true for other kernels in general.}   \hfill \\
($(1 + \bm{u} \cdot \bm{v})^2$ would take other forms for other cases).   \hfill \\
\hfill \\

It is essential that everything you want to do with your transformed vectors is representable by simple dot products.
If you can't simplify it down to dot products of the original vector then there's no point. 







\section{General Vocab}
\smallskip \hrule height 2pt \smallskip
 
 \begin{itemize}
 	\item \textbf{classification} - ??? Finding a f that converts X to Y where Y are categorical.  (Not regression).  
	\item \textbf{supervised learning} - at training time we are given a set of features with discrete class labels.  % Wk 4 audio transcription.
 	\item \textbf{held-out data}: 
			the terms "held-out" and "validation" are usually synonymous  % 2/25/2015 class forum (my question)
	\item \textbf{hypothesis space}: ?  E.g. binomial distribution for coin flip.  	
 	\item \textbf{prediction error}: measure of fit (?) 
	\item \textbf{regularization}: a process of introducing additional information in order to solve an ill-posed problem or to prevent overfitting  % https://en.wikipedia.org/wiki/Regularization_(mathematics)
	\item \textbf {norm}.  A scalar (not vector!) measure of distance between two vectors.  % E confirmed 3/6/2016
					You can divide by this scalar to \underline{norm}alize, but that is just one use case.
					You can also use either norm as a regularization term. 
	\item \textbf{L1} norm.  Just add the absolute values of the components.  % E confirmed 3/6/2016
	\item \textbf{L2} norm.  Can be used to regularize, or to normalizing features. 
		\begin{itemize}
			\item Euclidean vector length.  Pythagoras style. 
			\item You can do $l_2$ normalization for a feature vector to get a unit vector: 
				Convert $x$ to $\hat{x}$ so that if you form $||\hat{x}||_2^2 = 1$
				Can also do $l_1$
		\end{itemize}
	\item \textbf{L2} distance:  Pythagoras distance between two vectors. 
	\item \textbf{convergence} - if you add more data and you don't get different parameters, the model has converged. 
	\item \textbf{kernel}: some transformation of your features that improves your classification %Erick 1/30/2015
	%\item \textbf{}
	\item \textbf{affine}: indicates that the subspace need not pass through the origin.  % Intro to statistical learning Ch 9.1
	\item \textbf{support vector}: data points that �support� the maximal margin hyperplane in the sense that if these points were 				moved slightly then the maximal margin hyper- plane would move as well.   % Intro to statistical learning Ch 9.1
	\item \textbf{marginal likelihood}: 
	\item \textbf{marginalized out} = integrated out % https://en.wikipedia.org/wiki/Marginal_likelihood

 \end{itemize}
 
 Concepts:
\textbf{likelihood vs posterior}: \hfill \\

	likelihood*prior = constant*posterior.  $P(Y|X)$ is likelihood, $P(X|Y)$ is posterior.  
\hfill \\
\hfill \\

\subsubsection{Normalizing data}
Options:
\begin{itemize}
	\item min/max
	\item sigmoid
	\item $L_1$ norm
	\item $L_2$ norm
\end{itemize}
Note $L_1$/$L_2$ normalization versus penalty/regularization!!  

\subsubsection{Size for modeling P$(Y=y | X)$}
		How many parameters are needed to model $P(y | x_1, x_2, ..., x_d)$?  
		Assume Y is discrete and all the $x_i$ are binary.
		You have d binary features, so you have $2^d$ probabilities in order to classify every possible input.   % Erick
		
		The number of parameters in the PDF of $P(Y=y | X)$ is $2^d$ (+ 1 for bias sometimes).  
		The size (\# of nodes??) of the conditional probability tree grows exponentially.
		% The tree is the tree of conditional probabilities, i'll bet  % Erick. 
		The data is just a table with Y and X columns.  
		
		We only need $2^d$ (or perhaps $2^{(d+1)}$) to get all possibilities specified.  
            	If each feature can take m values instead then we have $m^d$.  Not binary any more!
		The number of parameters can get very big but Naive Bayes can handle it.  \hfill \\
		
		What is the order of the size of the parameters you need to do the full conditional probability?
		For Naive Bayes it is d; \textbf{linear}.  And the full conditional would be exponential! 

\subsubsection{Maximum Likelihood Estimation (MLE)}

\textbf{Take log, take derivative, set equal to zero.} \hfill \\

Memorize. Likelihood is DATA. \hfill \\  % e-mail to self 2/1/2015 

The best mu for a gaussian is the mean. 
Maximized probability of this data being produced by the distribution. 
The data is most likely to be generated if the mean is mu. 
Did same for sigma.  We realized best way is to use the variance.  \hfill \\

\textbf{Wikipedia:}

To use the method of maximum likelihood, one first specifies the joint density function for all observations. For an independent and identically distributed sample, this joint density function is:

$$  f(x_1,x_2,\ldots,x_n\mid\theta) = f(x_1\mid \theta)\times f(x_2|\theta) \times \cdots \times  f(x_n\mid \theta). $$
  
Now we look at this function from a different perspective by considering the observed values $x1, x2, \dots, xn$ to be fixed "parameters" of this function, whereas ? will be the function's variable and allowed to vary freely; this function will be called the likelihood:


 $$  \mathcal{L}(\theta\,;\,x_1,\ldots,x_n) = f(x_1,x_2,\ldots,x_n\mid\theta) = \prod_{i=1}^n f(x_i\mid\theta) $$
  
Note that " ; " denotes a separation between the two input arguments: $\theta$ and the observations $x_1,\ldots, x_n$.

In practice it is often more convenient to work with the logarithm of the likelihood function, called the log-likelihood:


   $$ \ln\mathcal{L}(\theta\,;\,x_1,\ldots,x_n) = \sum_{i=1}^n \ln f(x_i\mid\theta) $$
  
or the average log-likelihood:

  $$ \hat\ell = \frac1n \ln\mathcal{L} $$
  
The hat over $\ell$ indicates that it is akin to some estimator. Indeed, $\hat{\ell}$ estimates the expected log-likelihood of a single observation in the model.

The method of maximum likelihood estimates $\theta_0$ by finding a value of $\theta$ that maximizes $\hat\ell(\theta;x)$. This method of estimation defines a maximum-likelihood estimator (MLE) of $\theta_0$:


    $$  \{ \hat\theta_\mathrm{mle}\} \subseteq \{ \underset{\theta\in\Theta}{\operatorname{arg\,max}}\ \hat\ell(\theta\,;\,x_1,\ldots,x_n) \} $$

	
\subsubsection{MLE vs MAP}
\begin{itemize}
		\item both MLE and MAP are point estimates.  No estimate of uncertainty.  % https://www.youtube.com/watch?
		\item MLE fits a probabilistic model $P(x | \theta)$ to data to estimate $\theta$. % http://www.cs.colostate.edu/~cs545/fall13/dokuwiki/lib/exe/fetch.php?media=wiki%3A13_naive_bayes.pdf
			You chose the parameters $\theta$ that maximize $\ln P(X | \theta)$
		\item MLE is more likely to overfit; MAP regularizes to prevent overfitting.  \hfill \\  % https://www.youtube.com/watch?v=kkhdIriddSI 
		\item MAP doesn't have all the nice asymptotic relationship, but tends to look like MLE asymptotically.
			As your data goes to infinity, your prior foes to the data.  % https://www.youtube.com/watch?v=kkhdIriddSI 
		\item unlike MLE, MAP is not invariant under reparameterization.  (A disadvantage)  % https://www.youtube.com/watch?v=kkhdIriddSI
		\item in MAP, you have to chose a prior.  Sometimes not fun to pick.  % https://www.youtube.com/watch?v=kkhdIriddSI
\end{itemize}

\underline{TA e-mail} \hfill \\
In both of these problems, we assume that we have some i.i.d. data $\{x_1, ... , x_M\}$ which was drawn from a distribution $P(x_i | \theta)$, and we want to estimate the parameters $\theta$).

In the MLE scenario, we directly optimize the (log) likelihood of the data $P(X | \theta)$, to get the estimate of $\theta$ under which the data has maximum likelihood.

In the MAP scenario, we also have some prior knowledge about what $\theta$ should be, e.g. $P(\theta) =$ Normal$(0, \sigma)$. We can then use Bayes' rule to get $P(\theta | X)$:

$$ P(\theta | X) = P(X | \theta) P(\theta) / Z$$
where Z is the normalizing constant

That is, we want our estimate of theta to: \hfill \\
1) Model the data well,  \hfill \\
2) Have a high prior probability.  \hfill \\

\textbf{Note that when $P(\theta)$ is constant, i.e. we don't have any prior knowledge, MAP just reduces to MLE.}

\subsubsection{How Naive Bayes, MLE, and MAP fit together}  % Erick approved.
In order to do Bayesian inference, you put your likelihood into Bayes rule.  That has nothing to do with Naive Bayes.
If you just maximize the ML equation, that gives you MLE.
If you maximize the posterior that you get from Bayes rule then you have MAP.  
All that is true across analysis.

A full Bayesian analysis for many classification problems is too hard.  Too parameter rich, too computationally expensive.
You can simplify by making the Naive Bayes assumption. 
In machine learning world you will usually do this simplification. 

Note: Erick doesn't think he will ever use Naive Bayes.  
He does statistical analysis, not Naive Bayes.  
Naive Bayes is a much smaller and specialized thing than general Bayesian analysis.
That's the statistician perspective. 

For ML, it is a nice way of gettitng regularized estimates. 
You could also use Bayes to relate parameters in a model. 
That's what Erick does a lot. 

\subsubsection{Smoothing}
Can reduce sensitivity to zero values when multiplying probabilities.
\begin{itemize}
	\item prior distribution for binaries: Beta distribution. 
	\item Laplace smoothing for multinomial. 
	
\textbf{Bayes b/c you aren't going to see a feature vector that matches one in training.}
	We are \textbf{not} going to see a feature that is the exact same as a feature in the training set. 
	That's \textbf{why} we estimate w/ Bayes.   % week 5 notes
\end{itemize}

\subsubsection{Generate vs. Discriminative}
\textbf{Discriminative tries to learn $P(y | x)$, whereas a generative model tries to learn $P(x,y) = P(y|x)P(x).$ }   \hfill \\

Since the generative model learns $P(x)$ as well as $P(y|x)$, it often requires much more data, but can do things like actually generating sampled data from scratch by drawing from $P(x)$ and then from $P(y|x)$ (hence the term 'generative' model). Given that, you could classify the algorithms above as follows:

Generative: 
\begin{itemize}
	\item Naive Bayes (since we learn class distributions $P(y_i)$ and models $P(x_ij | y_i)$, a bit backwards from above, but we're still learning a generative model)
\end{itemize}
Discriminative: 
\begin{itemize}
	\item Decision Trees
	\item Perceptron
	\item SVM
\end{itemize}

Point estimation and linear regression don't really fit as well into this framework, since we're usually talking about classifiers when making this distinction. For further info you can read the discussion here: \href{http://stats.stackexchange.com/questions/12421/generative-vs-discriminative}{Stats.stackexchange}
  \hfill \\

logistic: discriminative  \hfill \\
Naive Bayes:  generative   \hfill \\
\hfill \\

One can only distinct between whether it is something, the other can say how likely. \hfill \\

Two big categories of approaches for ML: \hfill \\
\underline{Generative}:
		Those that try to estimate the joint distributions between labels and features/data. 
		Model the joint distributions. $P(X,Y) = f(X|Y) P(Y) or P(Y|X)f(x)$.  
		"Class conditional densities" are modeled.  % https://www.youtube.com/watch?v=oTtow2Ui8vg 
		\textbf{If you can create a distribution, you can sample from it.}  % wk 5 transcript
		More powerful if you have enough data to estimate the densities, but worse without enough data.
		Natural interpretation.  
		Bayes classification is an example.    
		Naive Bayes is one that can produce p(Data,Zebra), except you've made a lot of assumptions.   \hfill \\  % Wed Wk 5 transcript
		P(Data,Zebra) is a pdf over samples. 
		If you didn't relax all the constraints when going from Bayes to Naive Bayes, you could paint a zebra.   \hfill \\
		A joint probability model with evidence variable.   \hfill \\  % Lec 7: preceptrons
 \hfill \\
\underline{Discriminative}:
		No generative model, no Bayes rule, often no probabilities at all!   \hfill \\  % Preceprtons (Lec 7)
		Those that directly estimate the decision boundary.  "discriminative decision boundary".  Find $P(Y|X)$ \hfill \\
		Describing how to generate random instances X conditioned on the target attribute Y.  \hfill \\
		The discriminative classifier is like a lazy painter.  \hfill \\  % Wed Wk 5 transcript
		You can get decision boundary out of a generative model but it is over-kill if you only want to produce a label.  \hfill \\  % Wed Wk 5 transcript
		 85\% of time discriminative outperforms.  B/c p(Data,Zebra) is hard to estimate.  \hfill \\  % Wed Wk 5 transcript
		 		 
\includegraphics[width=2.5in]{figures/zebra_painting.pdf}

Why does the generative p graph max out at 0.1 and the discriminative at 1?   
    Discriminating against zebra or not zebra.  Has to sum to 1.  
    The discriminative doesn't have to sum to 1.  Many other things have to sum with it to sum to 1. 
    
\underline{TA review:} \hfill \\
In discriminative models we just want to discriminate between classes given the data: learning $P(y_i | X_i)$. 
In generative models we are trying to learn the joint distribution $P(y_i, X_i) = P(y_i) P(X_i | y_i)$, which we can then use to determine $P(y_i | X_i)$ by Bayes rule. 
Although this requires more data since we're learning $P(X_i | y_i)$, these models can be much more interpretable than discriminative models.
 
 \subsection{Linear Classifiers}
 Inputs are feature values.  \hfill \\
 Each feature has a weight.  \hfill \\
 Sum is the activation.  activation$_w(x) = \sum_i w_i x_i = w \cdot x$  \hfill \\
 If the activation is positive, chose output class 1.  \hfill \\
 If the activation is netative, chose output class 2.  \hfill \\
 
 \includegraphics[width=1.5in]{figures/linear_classifier_cartoon.pdf}  \hfill \\
 
 For a binary decision rule:   \hfill \\
 In the space of feature vectors: 
 \begin{itemize}
 	\item examples are points
	\item any weight vector is a hyperplane
	\item one side corresponds to y = +1
	\item the other side corresponds to y = -1
	\item ??? The $w \cdot x = 0$ is the solution to the line.
 \end{itemize}
 
 \includegraphics[width=1.5in]{figures/binary_decision_rule.pdf}
 
 \subsection{NB, LR, Perceptron}
\includegraphics[width=2.5in]{figures/three_views_of_classification.pdf}

\subsection{Gradient Ascent/Descent vs Coordinate Ascent/Descent}
 We discussed gradient descent with logistic regression. \hfill \\
 We mentioned coordinate descent when getting ready to talk about EM.   \hfill \\

\subsubsection{Gradient descent}
For finding the local minimum. 
\begin{itemize}
	\item Takes steps proportional to the negative of the gradient 
		(or of the approximate gradient) of the function at the current point. 
	\item If instead one takes steps proportional to the positive of the gradient, 
		one approaches a local maximum of that function; 
		the procedure is then known as gradient ascent.
\end{itemize}	

\subsubsection{Coordinate descent}
\begin{itemize}
	\item A non-derivative optimization algorithm
	\item To find a local minimum of a function, one does line search along 
		one coordinate direction at the current point in each iteration. 
		One uses different coordinate directions cyclically throughout the procedure.
	\item Has problems with non-smooth functions 
		% https://en.wikipedia.org/wiki/Coordinate_descent
	\item Coordinate descent \textbf{does} have step size parameter. % week 10 audio
		To prevent over-shooting, you many need to take smaller steps. 
	\item  Does converge under two big assumptions: \hfill \\
		(1) fixing one works.   \hfill \\
		(2) need loss to get smaller than it was before. \hfill \\
	\item Won't always converge to global optima.
	\item For coordinate descent, you have to be extremely careful that each step reduces the loss function.
		If you can't prove that you should not use coordinate descent. 
\end{itemize}	

K-means does this: alternate between holding the assignments and the centers fixed. 






\vspace{4in}
\bigskip


\end{multicols*}
\end{document}
