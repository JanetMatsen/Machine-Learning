\section{Bayesian Learning}
\smallskip \hrule height 2pt \smallskip

Rather than estimating a single $\theta$, we obtain a  distribution over possible values of $\theta$.

For small sample size, prior is important! 

Use Bayes' Rule:
$ \displaystyle P(\theta | D) = \frac{P(D | \theta) P(\theta)}{P(D)}$
\begin{itemize}
	\item \textbf{Posterior}: $P(\theta | D)$
	\item \textbf{Data Likelihood}: $P(D | \theta) $
	\item \textbf{Prior}: $P(\theta)$
	\item \textbf{normalization}: $P(D)$
\end{itemize}
Or equivalently, $P(\theta | D) \propto P(D | \theta) P(\theta)$

If you have a uniform prior, you just do MLE.  \hfill \\
$P(\theta) \propto 1 \rightarrow P(\theta | D) \propto P(D | \theta)$

\underline{Vocab}
\begin{itemize}
	\item \textbf{prior}: 
	\item \textbf{prior distribution}: 
	\item \textbf{posterior}: 
	\item \textbf{posterior distribution}: 
	\item \textbf{MAP}:
\end{itemize}

\hfill \\
\underline{Thumbtack Problem}
\begin{itemize}
	\item use Binomial likelihood:  $P(D | \theta) = \theta^{\alpha_H} (1-\theta)^{\alpha_T}$
	\item To get a simple posterior form, use a conjugate prior.  Conjugate prior of Binomial is the Beta Distribution.  See \href{http://courses.cs.washington.edu/courses/cse446/16wi/Slides/3_PointEstimation.pdf}{slides} for math. 
	\item The Beta prior is equivalent to extra thumbtack flips.  As $N \rightarrow \infty$, the prior is �forgotten�.  But for small sample size, prior is important.  
\end{itemize}

If you are measuring a continuous variable, Gaussians are your friend. 